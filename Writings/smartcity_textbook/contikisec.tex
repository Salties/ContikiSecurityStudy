\chapter{Contiki OS}
Assuming this is the structure...

\section{Security in Contiki OS}

%Though these paragraphs might not be necessary...
The Contiki OS has some component implemented in the security aspect. Several factors need to be taken into account when selecting security components to be used in the application, including:

\begin{itemize}
\item The trade-off between security and performance, such as the overhead of bandwidth, energy consumption, etc. Usually a higher level of security comes with more reduction in performance.

\item The capability of the potential adversary. For example, passive eavesdropping is a common type of attack in a Wireless Sensor Networks scenario. Further more, if the device is exposed in an open environment then side channel attacks\footnote{Attacks that exploit physical metadata, e.g. power consumption, timing information, etc.} need also be taken into consideration. In some cases legitimate users can also considered malicious,  say users who try to tamper with their smart meter readings.

\item The hardware and software setup of the platform. For instance, the AES coprocessor provided in CC2538 platform provides a great performance in both computation time and energy consumption; the latency induced by ContikiMAC can make attacks exploit the packet timing information more difficult.
\end{itemize}

However, due to the constrained resources and variant applications, implementing security protocols poses great difficulties in Contiki OS as well as other embedded operating systems.

\subsection{Link Layer Security}
Link Layer Security, or LLSEC, is an 802.15.4 implementation in Contiki OS.
%Continue from here. 2016/02/01.


\subsection{DTLS}
