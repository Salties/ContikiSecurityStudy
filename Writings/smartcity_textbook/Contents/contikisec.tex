\chapter{Contiki OS}
Assuming this is the structure...

\section{Securities in Contiki OS}

%Though these paragraphs might not be necessary...
%The Contiki OS has some component implemented in the security aspect. Several factors need to be taken into account when selecting security components to be used in the application, including:
%
%\begin{itemize}
%\item The trade-off between security and performance, such as the overhead of bandwidth, energy consumption, etc. Usually a higher level of security comes with more reduction in performance.
%
%\item The capability of the potential adversary. For example, passive eavesdropping is a common type of attack in a Wireless Sensor Networks scenario. Further more, if the device is exposed in an open environment then side channel attacks\footnote{Attacks that exploit physical metadata, e.g. power consumption, timing information, etc.} need also be taken into consideration. In some cases legitimate users can also considered malicious,  say users who try to tamper with their smart meter readings.
%
%\item The hardware and software setup of the platform. For instance, the AES coprocessor provided in CC2538 platform provides a great performance in both computation time and energy consumption; the latency induced by ContikiMAC can make attacks exploit the packet timing information more difficult.
%\end{itemize}
%
%After all, due to the constrained resources and variant applications, implementing security protocols poses great difficulties in Contiki OS as well as other embedded operating systems.
%
%In this section, we will cover two major security component that has been implemented on Contiki OS, Link Layer Security and DTLS.

Implementing security protocols poses great difficulties in IoT devises due to the constrained resources and variant applications. In this section, we will cover two security components that has so far been implemented on Contiki OS, namely LLSEC and DTLS, respectively.

\subsection{LLSEC: noncoresec}
Link Layer Security, or LLSEC, is a security mechanism at Link Layer level. In Contiki OS, noncoresec is the 802.15.4 security instantiation that has been implemented. Its design goal is to provide:
\begin{itemize}
\item Data confidentiality over MAC payload.
\item Authenticity and integrity over MAC header and MAC payload.
\end{itemize}

noncoresec is disabled by default. When enabled, different security levels can be configured from no security, to encryption / authentication only, then to full encryption and authentication. The length of authentication tag is also configurable.

Briefly speaking, noncoresec is implemented by the following components:

\begin{description}
\item[\textbf{Block Cipher}] \hfill \\
As specified by 802.15.4 specification, AES\footnote{Advanced Encryption Standard\cite{AES}}-128 is chosen as the underlying block cipher\footnote{A block cipher takes a block of data of fixed length and a key and outputs a block of ciphertext.}. Contiki OS includes a software AES-128 implementation in its library. On those platforms with an AES coprocessor, such as CC2538, it can be switched to use the hardware implementation instead. The benefit for using hardware implementation is to have a better time and energy efficiency as well as to gain potential protections against some side channel analysis attacks\cite{DPA1}\cite{DPA2}, which will be discussed later.

\item[\textbf{Mode of Operation}] \hfill \\
Also specified by 802.15.4 specification, the AES block cipher is used in CCM* mode, i.e. CTR mode with CBC-MAC. The asterisk symbol implies the additional support for security levels and additional requirements to encode the security level into the nonce, which allows the recipient to uniquely deduce the format of an encrypted packet. The length of authentication tag is also specified by the security level, which is then imposed to the whole network.

\item[\textbf{Key Management}] \hfill \\
 A hard coded AES-128 key is shared among the whole network. This key will be used for both encryption and authentication under CCM* mode, for all incoming and outgoing data frames of every node.
 
 \item[\textbf{Nonce Management}] \hfill \\
 The 16 byte nonce is uniquely derived from the 802.15.4 header, including a flags field (1 byte), a source address field (8 byte), a frame counter field (4 byte), a security level filed (1 byte) and a 2 byte block counter which increases by 1 for each block of data.
 
\item[\textbf{Replay Protection}] \hfill \\
noncoresec has implemented the replay protection by comparing the received frame counter with the last frame counter from the same source.
\end{description}

The implementation is straight forward and relatively simple. In general, noncoresec has the following benefits:
\begin{enumerate}
\item \textbf{An eavesdropper cannot observe the plaintext of Link Layer payload.}
\item \textbf{Illegal nodes cannot joining the network,} as nodes without the knowledge of the network shared key cannot send or receive any RPL message.
\item \textbf{It can be implemented efficiently} on most platforms, especially with hardware support.
\item \textbf{Multicast and broadcast is supported.} This is a result by the nature of 802.15.4.
\end{enumerate}

However, the following factors should also be taken into account  when adopting noncoresec as the security measure:
\begin{enumerate}
\item \textbf{Lack of flexibility.} The access to the network is either no access or full access. There is no key pair or group key supported.
\item \textbf{Fixed key.} There is yet no key updating scheme implemented. 
\item \textbf{Reused nonce.} Since in CCM mode, the difference of two ciphertext is exactly the same of their according plaintext and knowing that can lead to breach of data confidentiality in many cases. The reuse may occur under many circumstances, such as when the 4 bytes frame counter rounds up, or when the device reboots and resets the frame counter back to $0$.
\end{enumerate}

\cite{802154sec} gives a thorough security analysis of 802.15.4 security.

\subsection{DTLS: tinydtls}
DTLS is derived from the widely used TLS protocol on Internet. As of TLS, DTLS also provides encryption and authentication between two nodes. It is currently adopted as the security measure of CoAP.

The main difference between TLS and DTLS is that TLS is based on TCP whilst DTLS is based on UDP. In IoT applications, UDP is generally preferable than TCP due to its lightweight design and thereby is DTLS  than TLS. In addition, DTLS also provides a simple reliable retransmission mechanism as a result for providing data integrity.

DTLS is a stateful session based protocol. This implies that:
\begin{itemize}
\item A handshake must be performed between two nodes before any data can be transmitted.
\item Different session keys are derived for each session during the handshake.
\end{itemize}

In Contiki OS, DTLS is provided by a third party implementation named tinydtls\cite{tinydtls}. The current version of tinydtls supports two cipher suites:

\begin{description}
\item[\textbf{TLS\_PSK\_WITH\_AES\_128\_CCM\_8}] \hfill \\
The DTLS server and client uses a pre-shared master secret value. The session key, an AES-128 key, is then derived from the master secret alongside with two random values exchanged during the handshake. The application data is then encrypted with AES-128 block cipher in CCM mode with 8 byte authentication tag. No certificate verification is performed when using this cipher suite.
\item[\textbf{TLS\_ECDHE\_ECDSA\_WITH\_AES\_128\_CCM\_8}] \hfill \\
When this cipher suite is used, the DTLS server and client performs an ECDHE\footnote{Elliptic Curve Diffle Hellman key Exchange} key agreement using public keys and parameters signed by ECDSA\footnote{Elliptic Curve Digital Signature Algorithm}. The session key, an AES-128 key, is an output of ECDHE key agreement. The application data is then encrypted with AES-128 in CCM mode with 8 byte authentication tag thereafter.
\end{description}

Comparing to LLSEC, DTLS has several advantages:
\begin{enumerate}
\item \textbf{Flexibility and scalability}. DTLS is built upon UDP, an application can therefore choose whether tp or not to use encryption thus avoids overhead. It also allows multiple connections being established between two nodes for different purposes, or different parameters being used for different nodes. Further more, a DTLS connection can be established dynamically between two nodes that are not linked locally, in contrast to LLSEC that only works between nodes that are directly connected.
\item \textbf{Dynamic key management}. DTLS connections are established during running time. The applications can dynamically update the keys and ciphers on desire.
\item \textbf{Interoperability}. DTLS is implemented above Transportation Layer (TCP/UDP) and therefore Physic Layer and Link Layer details are eventually transparent to the applications. This feature is significantly important when two nodes are sited in different types of networks, such as accessing a wireless sensor node in a 6LowPAN network from a client running on a desktop. 
\item \textbf{End to end security}. Since session keys are generated independently for each DTLS session, the corruption of one node in the network does not affect other nodes. In comparison, the breach of the network shared key of noncoresec will directly result into revelation of all plaintext among the whole network.
\item \textbf{Additional reliability}. Even though UDP is unreliable, DTLS implements a simple retransmission and window mechanism in order to provide data integrity.
\end{enumerate}

Unfortunately, these advantages do not come free. Several drawbacks of using DTLS need also be noticed:

\begin{enumerate}
\item \textbf{Additional resource requirement.} e.g. compiling with tinydtls library on CC2538 platform costs about 80KB additional code size. It also requires more memory during running time as DTLS is a stateful protocol and thus the state of each session needs to be preserved in memory.
\item \textbf{Bandwidth overhead.} The DTLS header consumes about 30 bytes of bandwidth for each packet. Considering the fact that the the minimum MTU\footnote{Maximum Transmission Unit} required by 6LowPAN is only 127 byte; the 30 byte overhead can be very impactful to some bandwidth intense applications. 
\item \textbf{Exceeding length of handshake packets.} DTLS handshake messages are usually longer than normal data packets. Specifically in tinydtls, the length of the longest handshake packets of 
TLS\_PSK\_WITH\_AES\_128\_CCM\_8 \\
and \\
TLS\_ECDHE\_ECDSA\_WITH\_AES\_128\_CCM\_8 are about 150 and 200 byte respectively. These packets can easily exceed the PMTU\footnote{Path Maxmimum Transmission Unit} in constrained environments and hence being dropped, resulting into failed handshake and waste of energy as well as bandwidth.
\item\textbf{Low performance.} Some cryptographic operations, such as curve computations, takes extraordinary long time to perform. For instance, a \\ TLS\_ECDHE\_ECDSA\_WITH\_AES\_128\_CCM\_8 handshake requires about 80s to complete on a CC2538 platform. The problem can aggravate to disfunctioning in some cases combined with other factors. For example, when an user tries to access a sensor node from his desktop through CoAPS\footnote{CoAP with DTLS}, the handshake can hardly be completed. The reason is that the desktop completes its computation immediately and thus will rapidly send out multiple exceedingly length handshake packets. Whereas the sensor cannot respond to the packets accordingly due to its constrained computational power.  Eventually some packets will be dropped due to the limited buffer size of sensor node.
\item \textbf{No multicast support (yet).} Multicast poses a great challenge to key management and cryptographic protocol design, especially in the constrained devices. Not being able to cooperate with IPv6 multicast will result into severe reduction of functionality, e.g. the multicast feature of CoAP. Some attempts\cite{multicast1}\cite{multicast2} has been made, but there is yet a concrete solution to this problem.
\end{enumerate}

In conclusion, an application designer needs to carefully balance the trade off between security and its cost before adopting DTLS as the security measure.

\subsection{Other security concerns}
The security components we described above in Contiki OS do not provide any countermeasure against side channel attacks. For example, neither LLSEC nor DTLS provides any mechanism to hide  metadata of packets, such as length, headers or timing information. These features can potentially be exploited by a traffic analysis attacker\cite{ta1} \cite{ta2} \cite{ta3} to obtain knowledge of the application data.  Also the built in naive AES-128 software implementation of Contiki OS does not apply any power analysis\cite{DPA1} \cite{DPA2} countermeasures; thus devices with a general processor could be vulnerable targets to these types of attacks when deployed in an exposed environment.

\subsection{Conclusion}
In this section, we reviewed two major security components that has been implemented for Contiki OS by briefly discussing their features alongside pros and cons. 

noncoresec is a LLSEC method and it implements 802.15.4 security with a network shared key on Contiki OS. It is efficient and protects application data from a low level but also lacks flexibility. Besides, its key and nonce managements are potentially risky.

DTLS is another method to implement security on Contiki OS. Comparing to LLSEC, DTLS provides end to end security and is more flexible and scalable. It also can better interoperate between different networks. The disadvantages of DTLS are mostly due to intensive resource requirements and lack of support of multicast.

Finally, we discussed some potential side channel attacks that cannot be simply prevented by these components and hence requires additional care during deployment.

\begin{thebibliography}{99.}
\bibitem{AES} NIST (2001) Announcing the ADVANCED ENCRYPTION STANDARD (AES). Available via\\
\url{http://csrc.nist.gov/publications/fips/fips197/fips-197.pdf}

\bibitem{802154sec} Sastry N, Wagner D (2004) Security Considerations for IEEE 802.15.4 Networks. Proceedings of the 3rd ACM Workshop on Wireless Security 32--42

\bibitem{tinydtls} tinydtls. \\ 
\url{http://tinydtls.sourceforge.net/}

\bibitem{multicast1} DICE Working Group (2015) DTLS-based Multicast Security in Constrained Environments
 draft-keoh-dice-multicast-security-08 (Expired Internet-Draft (individual)). Available via\\
\url{https://tools.ietf.org/pdf/draft-keoh-dice-multicast-security-08.pdf}

\bibitem{multicast2} Marco Tiloca (2014) Efficient Protection of Response Messages in DTLS-Based Secure Multicast Communication. Proceedings of the 7th International Conference on Security of Information and Networks, Glasgow, Scotland, UK, September 9-11, 2014 Page 466

\bibitem{ta1} Shuo Chen, Rui Wang, XiaoFeng Wang and Kehuan Zhang (2010) Side-Channel Leaks in Web Applications: {A} Reality Today, a Challenge Tomorrow. 31st {IEEE} Symposium on Security and Privacy, S{\&}P 2010, 16-19 May 2010, Berleley/Oakland, California, {USA} Page 191--206

\bibitem{ta2} Luke Mather and Elisabeth Oswald (2012) Pinpointing side-channel information leaks in web applications. J. Cryptographic Engineering 2:3 161--177

\bibitem{ta3} Dyer, Kevin P. and Coull, Scott E. and Ristenpart, Thomas and Shrimpton, Thomas (2012) Peek-a-Boo, I Still See You: Why Efficient Traffic Analysis Countermeasures Fail. Proceedings of the 2012 IEEE Symposium on Security and Privacy 332--346

\bibitem{DPA1} Stefan Mangard and Elisabeth Oswald and Thomas Popp (2007) Power analysis attacks - revealing the secrets of smart cards

\bibitem{DPA2} Colin O'Flynn and Zhizhang Chen (2015) Power Analysis Attacks against IEEE 802.15.4 Nodes. IACR Cryptology ePrint Archive 529\\
\url{http://eprint.iacr.org/2015/529.pdf}
\end{thebibliography}