\chapter{Potential Leakage Sources}

In this chapter, we discuss the potential leakage sources in the experimental environment we have set up in \Cref{Chp: Experiment Setup}. 

\section{A Review of Review} \label{Sec: A Review of Review}
%What is ``Information''

The first and fundamental question that must be clarified in this project is: what is ``information''?

If we review those attacks described in \Cref{Chp: Literature Review}:

\begin{itemize}
	\item In literatures about the flaws in implementations of algorithms and protocols, ``information'' is the plaintext, usually represented as a numerical value, such as \cite{802154sec} \cite{rfc7457} \cite{CompressionRatioAttack} \cite{PaddingOracle}. Or it could be the secret key materials in many cryptographic side channel attack literatures, such as \cite{DPA}.
	\item In Traffic Analysis Attacks against web applications, ``informaiton'' is the user input, such as \cite{PinpointWeb} \cite{SearchAttack}, or the content of websites such as \cite{WebSideChannel}.
	\item When we talk about website fingerprinting such as \cite{WebsiteFingerprint} \cite{Peekaboo} and \cite{PClassifier}, ``information'' is the identity of websites. 
	\item The category only gets even more messy and trivial when we look at other Traffic Analysis Attacks, such as the spoken language and phrases in \cite{VoIPLanguage} and \cite{VoIPPhrases}, user event and OS of Apple product in \cite{AppleMessage} or even the ambient change and motion event in \cite{Video}.
\end{itemize}

``Information'' is always a relative concept to ``application'', and so is ``leakage''. 

In this report, we are interested on leakages with respect to the following ``information'':

\begin{description}[style=nextline]
	\item[Content and Size of Application Data]
	The implication of Application Data varies on different applications. Notice that breaking the cryptographic primitive is not part of the goals in this project.
	
	\item[Cryptographic Key]
	This represents the corresponding key materials in the security measure.
	
	\item[Application Code Routine]
	This indicates the application specific feature.
	
	\item[Network Topology]
	This implies how Sensor Nodes connect to each other.
\end{description}

Notice that the information leakage is highly application specific. In this report, we discuss only the applications described in \Cref{Sec: Applications}. 

\section{Observables} \label{Sec: Observables}

Generally speaking, observables in captured traffic can be classified into two categories:
\begin{enumerate}
	\item Implicit observables. These imply the data an adversary can access without inspecting the content of packets.
	\item Explicit observables. These imply the data given in the unencrypted part of packets.
\end{enumerate}

\subsection{Implicit Observables}

There are mainly two observables we concern in this category:

\begin{description}[style=nextline]
	\item[Packet Size] 
	This refers to the size (or length) of a packet. This value is also explicitly included in the packet headers.
	
	\item[Packet Timing]
	This refers to the time being sent of a packet. 
\end{description}

In addition to Packet Size and Packet Timing, there are some other implicit observables we do not concern in this report:

\begin{description}[style=nextline]
	\item[Presence of Packets]
	Apparently, the presence of a packet implies the existence of a Sensor Node device, which is kind of an information leakage. However, we do not take this into account. We assume the adversary has prior knowledge of the existence of WSN.
	
	\item[RSSI]
	Receive Signal Strength Indicator, RSSI, indicates the RF signal strength from a Sensor Node. Practically speaking, by measuring the RSSI of frames sent from a specific source MAC address, the adversary is capable to reveal the physical location of Sensor Node. In this report we do not take RSSI into account as its physical character is beyond the scope of this project. We assume the adversary has prior knowledge of the geographic information of the Sensor Nodes.
\end{description}

The implicit observables are commonly accessible in all scenarios in our experiments.

\subsection{Explicit Observables}

We have explained in \Cref{Chp: Experiment Setup} that there are two security measures implemented on our platform, namely noncoresec and DTLS. Referring to \Cref{Chp: Building Blocks}, these measures are implemented at different layers. The observables are therefore different for noncoresec and DTLS. 

Notice that we do not attempt to break the cryptographic primitives in this research; therefore we assume the value of ciphertext to be a random number and contains no information of its plaintext, except potentially the length.

\subsubsection{noncoresec} \label{Subsubsec: Explicit noncoresec}

As explained in \Cref{Subsec: 802.15.4 Security Implementation in Contiki}, noncoresec is the 802.15.4 Security implementation on Contiki. When noncoresec is enabled, all captured frames is in the form we explained in \Cref{Subsec: 802.15.4 MAC} and \Cref{Subsec: 802154 Sec}. The MAC Layer Header is the only observable part of a packet.

We stated in \Cref{Subsec: noncoresec in experiment} that we always use highest Security Level of 802.15.4 Security; therefore:
\begin{itemize}
	\item MAC Payload is always authenticated and encrypted in AES-128 with CCM* mode.
	\item Security Level is constantly 0x7.
	\item Key Strategy is constantly 0x0 as noncoresec does not implement any key management.
\end{itemize}

As a result of MAC Payload being authenticated and encrypted, 
\begin{itemize}
	\item The adversary cannot join the 6LoWPAN network as we have explained in \Cref{Subsec: noncoresec in experiment}.
	\item The adversary cannot distinguish an IPv6 packet and an ICMPv6 packet by simply looking the contest as it is encrypted. 
\end{itemize}

The IPv6 packets effectively are the packets with Application Data. ICMPv6 packets in our experiments are basically RPL messages.

\subsubsection{DTLS} \label{Subsubsec: Explicit DTLS}

As explained in \Cref{Chp: Building Blocks}, when DTLS is used, the explicit observables are: 

\begin{itemize}
	\item MAC Header as explained in \Cref{Subsec: 802.15.4 MAC}.
	
	\item Compressed IPv6 Header as explained in \Cref{Subsec: 6LoWPAN Adaptation Sub Layer}, \Cref{Subsec: IPv6 Data Packets} and \Cref{Subsec: ICMPv6}.
	
	\item UDP Header as explained in \Cref{Subsec: UDP}.
	
	\item DTLS Header as explained in \Cref{Subsec: DTLS}.
\end{itemize}

Since DTLS is built on Application Layer, therefore it does not impose any setting to the lower layer headers. The only exception is the Protocol Version of DTLS Header since tinydtls supports only DTLS version 1.2, therefore Protocol Version is constantly \{0xfe, 0xfd\}, i.e. \{0xff, 0xff\} - \{0x01, 0x02\}.

\subsection{Traces} \label{Subsec: Traces}

Before discussing the ``traces'' in WSN applications, as a comparison, we review the different definition of ``traces'' in the literatures we reviewed in \Cref{Chp: Literature Review}:

\begin{itemize}
	\item In literatures about Traffic Analysis Attacks against web applications\cite{WebSideChannel}\cite{PinpointWeb}\cite{SearchAttack}, a ``trace'' refers to the continuous\footnote{We would argue that the term ``continuous'' is ambiguous; nevertheless it is the best definition we can thought of.} packets triggered by an user input.
	\item In Website Fingerprinting literatures\cite{WebsiteFingerprint} \cite{HClassifier} \cite{PClassifier} \cite{Peekaboo}, a ``trace'' refers to the continuous packets that triggered by a browser requesting a website.
	\item Other attacks use more specific application dependent definitions of a ``trace''. \cite{VoIPLanguage} and \cite{VoIPPhrases} defined a ``trace'' to be the continuous packets during a VoIP conversation. \cite{Video} defines it to be the continuous packets during a video conversation. \cite{AppleMessage} does not even use the term ``trace'' at all\footnote{Actually it did used once as a verb equivalent to ``track''.} as it only analyses a single packet.
\end{itemize}

Practically, the ``trace'' of an Internet application usually can be isolated by filtering packets belong to the same TCP connection, i.e. TCP packets with the same entities\footnote{An entity is the combination of an IP address and a port}. However, we cannot apply the same method on many WSN applications due to the change of protocol suite. Further more, the WSN applications perform Machine To Machine, M2M, communication which has much less external intervention, makes it harder to group packets into ``traces''.

Another factor needs to be taken into concern is the different characteristics of WSN applications and Internet applications. As a result of resource constrained environment, that WSN application designs tend to be simple and stateless, that is to say the computation on the Sensor Node needs to be simple enough to be performed on such devices and data transmission is minimised. From this aspect, we argue that the simple applications we described in \Cref{Sec: Applications} captured these characteristics of WSN applications.  

We explicitly define a model for our WSN applications.

\begin{definition} \label{Def: Session}
A Session is defined as a series of events where a Sensor Node reports Application Data to a Manager. A Session includes an optional Request sent from Manager to Sensor Node, and a Response containing the Application Data sent from Sensor Node to Manager.
\end{definition}

\begin{definition} \label{Def: Trace}
A Trace is defined as all captured packets within one Session.
\end{definition}

In our applications described in \Cref{Sec: Applications}, there are typically only two packets in a trace. The first one corresponds to the Request and the second corresponds to Response. In case of other applications, e.g. the broadcast keyllsec application in \Cref{Sec: Applications} or a CoAP application using OBSERVE option as we explained in \Cref{Subsec: CoAP}, there might be only one Response packet in a trace without a Request.

As of this project we consider only information leakage within single Session. One might argue that there is potential information leakage when jointly analyse traces from different Sessions, such as the frequency of Requests can ``leak'' itself. We do not take these leakage into concern as it does not make much sense without considering within a higher layer application. Also in our experiments, we make a reasonable assumption based on our code that each Session is independent, as each execution of the code is independent to its last execution.

%Further more, in our applications the content of Request is a constant 3 bytes ASCII string ``GET''. We assume this known by the adversary. We make this assumption to focus on the leakage of the Application Data in Responses. 

\section{Theoretical Analysis}

In this section we provide a theoretical analysis over all observables we explained in \Cref{Sec: Observables}. It is done in two approaches:

\begin{enumerate}
	\item Cross reference the intersection of the observables we explained in \Cref{Sec: Observables} with the known exploitable side channel information we introduced in \Cref{Sec: Traffic Analysis Attacks}
	\item Semantically analyse the relation between Application Data and observables. 
\end{enumerate}

We discuss only the leakage over the attributes we specified in \Cref{Sec: A Review of Review}.

For some of the observables and attributes, we argue that:

\begin{theorem} \label{Te: IR}
For the definitions in \Cref{Subsec: Information Theory}, if the observable $Y$ is independent to the secret $X$, then there is $0$ leakage of $X$ over $Y$.
\end{theorem}

\begin{proof}
	%Independent random variables does not leak.
	Since $X$ and $Y$ are independent, therefore
	\begin{eqnarray*}
		\begin{aligned}
			P(x,y) &= P(x)P(y) \\
			P(x|y) &= P(x)
		\end{aligned}
	\end{eqnarray*}

	For Mutual Information and Capacity, we have:
	\begin{eqnarray*}
		\begin{aligned}
			H(X|Y) 
			&= - \sum_{x \in X} \sum_{y \in Y} P(x,y)\log{P(x|y)} \\
			&= - \sum_{x \in X} \sum_{y \in Y} P(x)P(y)\log{P(x)} \\
			&= \sum_{y \in Y} P(y) (- \sum_{x \in X}P(x)\log{P(x)}) \\
			&= \sum_{y \in Y} P(y) H(X) = H(X) \sum_{y \in Y}{P(y)} \\
			&= H(X)
		\end{aligned}
	\end{eqnarray*}
	
	Therefore
	\begin{eqnarray*}
		\begin{aligned}
			I(X;Y) &= H(X) - H(X|Y) = H(X) - H(X) = 0 \\
			C &= \sup_{\forall P(X)} I(X;Y) = \sup_{\forall P(X)} 0 = 0
		\end{aligned}
	\end{eqnarray*}
	
	Similarly for gain function based leakage\cite{GLeakage},
	\begin{eqnarray*}
		\begin{aligned}
			V_{g}(\pi, C) 
			&= \sum_{y \in Y}{\max_{w \in W}\sum_{x \in X}{\pi[x]C[x,y]g(w,x)}} \\
			&= \sum_{y \in Y}{\max_{w \in W}\sum_{x \in X}{\pi[x]P(y|x)g(w,x)}} \\
			&= \sum_{y \in Y}{\max_{w \in W}\sum_{x \in X}{\pi[x]P(y)g(w,x)}} \\
			&= \sum_{y \in Y}p(y){\max_{w \in W}\sum_{x \in X}{\pi[x]g(w,x)}} \\
			&= \max_{w \in W}\sum_{x \in X}{\pi[x]g(w,x)} = V_{g}(\pi)
		\end{aligned}
	\end{eqnarray*}
	
	Therefore
	\begin{equation*}
		H_g(\pi, C) = -\log{V_g(\pi, C)} = -\log{V_g(\pi)} = H_g(\pi)
	\end{equation*}
	
	Hence 
	\begin{eqnarray*}
		\begin{aligned}
			L_g(\pi, C) &= H_g(\pi) - H_g(\pi,C) = H_g(\pi) - H_g(\pi) = 0\\
			ML_g(C) &= \sup_{\pi} L_g(\pi, C) = \sup_{\pi} 0 = 0
		\end{aligned}
	\end{eqnarray*}
\end{proof}

\begin{corollary} \label{Cor: Constant Leakage}
	An observable with a constant value has no leakage under the definitions in \Cref{Subsec: Information Theory}.
\end{corollary}

\begin{proof}
	This directly followed by the fact that a fixed observable $y$ with $P(y) = 1$ is independent to the secret $X$.
\end{proof}

\subsection{Cross Reference of Observables with Known Attack} \label{Subsec: Cross Reference}

Despite the differences in the nature applications, we discuss the applicability of the packet features those have been exploited in Traffic Analysis Attacks, as we have summarised in \Cref{Tbl: Classifiers in Traffic Analysis Literatures}.

\begin{description}[style=nextline]
	\item[Direction]
	For a two packets Session, it is predictable that the first Request packet is from Manager to Sensor Node and the second Response packet from Sensor Node to Manager. In a one packet Session there is only one packet from Sensor Node to Manager.
	
	\item[Length]
	The same feature can be observed as an implicit observable.
	
	\item[Frequency Distribution of Length]
	The same feature can be computed by packet length. However, since there are typically only two packets in a trace, the result is likely to be $0.5$ for the length of Request packet and $0.5$ for the length of Response packet. In a one packet Session there is only one value in the distribution with probability of $1$.
	
	\item[Size, HTML and Number Markers]
	In a two packet Session there is only one direction change in a trace; thus the marker are constant marks the second packet. In one packet Session this feature is simply not applicable.
	
	\item[Total Bytes]
	The same feature can be computed through packet lengths.
	
	\item[Percentage Incoming Packets]
	The term ``incoming'' refers to the direction of web server to the browser in its original Web Fingerprint literature. In our experiments we assumed the adversary monitors all packets in the network; thus there is not an explicit definition of ``incoming'' and ``outgoing''. Further more, in our applications, there is only two packets in a Session for each direction, or only packet from Sensor Node to Manager.
	
	\item[Number of Packets]
	As explained in \Cref{Subsec: Traces}, this value is either $1$ or $2$ in our experiments, depends on the application running. 
	
	\item[Total Time]
	In a two packet Session this is exactly the interval between Request packet and Response packet. In a one packet session this is not applicable.
	
	\item[Total Per-direction Bandwidth]
	Since there is at most only one packet at each direction, this feature is effectively a single packet size divided by total time for each direction.
	
	\item[Traffic Burst]
	Traffic burst is reduced packet size in our applications as there is at most only one packet each direction.
\end{description}

As we can see from the description above, several packet features are fixed or not applicable in our applications, including:

\begin{itemize} 
	\item Direction
	\item Frequency Distribution of Length
	\item Size, HTML and Number Markers
	\item Percentage Incoming Packets,
	\item Number of Packets
\end{itemize}

According to \Cref{Cor: Constant Leakage}, we consider these packet features as non leakage features. Although arguably these features may be exploited to leak information of which application is running over the Sensor Nodes, in this report, we assume the adversary has this prior knowledge and hence do not consider such as an information leakage.

As a result, the potential leakage sources from the literatures are:
\begin{itemize}
	\item Length
	\item Total Bytes
	\item Total Time
	\item Total Per-direction Bandwidth
\end{itemize}

Notice that we ignored Traffic Bursts since it is reduced to packet length in our applications as explained above.

\subsection{Semantic Analysis}

In this section, we provide a semantic analysis over the observables based on their semantics we explained in \Cref{Chp: Building Blocks}.

In this report,  we assume the adversary has all identity information in the WSN and thus exclude the address information from potential leakage sources. Theoretically speaking, any valid address may be used in any packet and thus could be considered independent to the application but in practice, addresses are associated to specific Sensor Nodes in a WSN; therefore is a potential leakage source to the entity identities in a Session. 

Also notice that packets can be observed even if no application is running. These packets  are generated by the Contiki kernel to maintain the connectivity of the 6LoWPAN network, such as RPL packets. Since the existence of these packets is independent to the application running over WSN, we consider them as no leakage due to \Cref{Te: IR}. One exception is the DAO message we explained in \Cref{Subsec: ICMPv6} which could be triggered by the first application packet if the routing is not queried before. In this report, we consider the DAO message independent to the application.

As we have explained in \Cref{Chp: Building Blocks}, the 6LoWPAN protocol suite is constructed layer by layer and each layer is conceptionally transparent to each other. As a result, most of the headers are in theory independent to the application and thus are non leakable due to \Cref{Te: IR}. For these observables, we expect to see them distributed independently among different applications. In this section, we provide a theoretical analysis by discussing their link to upper layer applications based on their semantics explained in \Cref{Chp: Building Blocks}.

\subsubsection{Implicit Observables}

The implicit observables are indeed covered by the potential leakage sources we analysed in \Cref{Subsec: Cross Reference}. As a matter of fact, these observables turned out to be the most exploitable features in the literatures we have reviewed in \Cref{Sec: Traffic Analysis Attacks}.

%Linear relationship of length
In our experiments, we noticed that in our security measures, namely noncoresec and DTLS, the underlining cryptographic primitives are both AES-128 with CCM mode where no padding is applied. This effectively indicates that the length of plaintext and ciphertext are expected to have a linear relationship with slope ratio $1$:

\begin{equation} \label{Eq: Linear Length}
	l_{C} = l_{P} + b
\end{equation}

where $l_{C}$ and $l_{P}$ are the length in bytes for packets with and without encryption, and $b$ is a predictable constant induced by security measures such as MIC and change of headers. Practically when the security measure increases the packet length to a value exceeding MTU, the packet will be immediately dropped and cannot be observed by an adversary. However, considering the functionality of WSN, we assume the plaintext is always in a reasonable length that does not result into a ciphertext exceeding MTU.

%Leakage with linear relationship
Modelling the leakage of packet length as a channel $C(l_{C},l_{P})$ as in other Information Theoretic approaches we described in \Cref{Subsec: Information Theory}, we have a deterministic channel such that:
\begin{equation}
	C(l_{P}, l_{C}) = P(l_{P} | l_{C}) = 
	\begin{cases}
		1 &\text{if: } l_{C} = l_{P} + b \\
		0 &\text{otherwise}
	\end{cases}
\end{equation}

and

\begin{equation}
	C^{-1}(l_{C}, l_{P}) = P(l_{C} | l_{P}) = 
	\begin{cases}
		1 &\text{if: } l_{C} = l_{P} + b \\
		0 &\text{otherwise}
	\end{cases}
\end{equation}

So

\begin{equation}
	P(l_{P} , l_{C}) = P(l_{P}) P(l_{C} | l_{P}) =
	\begin{cases}
		P(l_{P}) &\text{if: } l_{C} = l_{P} + b \\
		0 &\text{otherwise}
	\end{cases}
\end{equation}

Therefore\footnote{Information Theory defines $0\log{0} = 0$.},
\begin{equation}
	P(l_{P} , l_{C}) \log{P(l_{P} | l_{C})} = 
	\begin{cases}
		P(l_{P})\log{1} = 0 &\text{if: } l_{C} = l_{P} + b \\
		0 \log{0} = 0 &\text{otherwise}
	\end{cases}
\end{equation}

Hence
\begin{equation}
	H(L_{P} | L_{C}) = - \sum_{l_{C} \in L_{C}} \sum_{l_{P} \in L_{P}}P(l_{P} , l_{C}) \log{P(l_{P} | l_{C})} = - \sum_{l_{C} \in L_{C}} \sum_{l_{P} \in L_{P}} 0 = 0
\end{equation}
where $L_{P}$ and $L_{C}$ are the possible length in bytes of encrypted and unencrypted packets.

In this case, the Mutual Information is:
\begin{equation} \label{Eq: MI in length}
	I(L_{P};L_{C}) = H(L_{P}) - H(L_{P} | L_{C} ) = H(L_{P}) - 0 = H(L_{P})
\end{equation}

For the Capacity, according to \Cref{Eq: MI in length}, $I(L_{P};L_{C})$ has its maximum value when $L_{P}$ is uniformly distributed:
\begin{equation} \label{Eq: Cap in length}
	Capacity = \sup_{\forall L_{P}}{I(L_{P};L_{C})} = \sup_{\forall L_{P}}H(L_{P}) = - \sum_{i = 1}^{|L_{P}|}|L_{P}|^{-1}\log{|L_{P}|^{-1}} = \log{|L_{P}|}
\end{equation}

In another word, \Cref{Eq: MI in length} and \Cref{Eq: Cap in length}  imply that averagely all bits of $l_{P}$ is leaked through $l_{C}$.

For the gain function based leakage\cite{GLeakage}, we realised that it would be hard to quantify the leakage without a specific gain function. Therefore instead, we provide an analysis with min-leakage.

In this case, the Posterior Vulnerability is:
\begin{equation}
	\begin{aligned}
		V(\pi_{L_P}, C^{-1}) 
		&= \sum_{l_{C} \in L_{C}} \max_{l_{P} \in L_{P}} \pi_{L_P}[l_P]C^{-1}[l_P,l_C] \\
		&=  \sum_{l_{C} \in L_{C}} P(l_C) \max_{l_{P} \in L_{P}} P(l_P | l_C) \\
	      &= \sum_{l_{C} \in L_{C}} P(l_C) = 1 \\
	\end{aligned}
\end{equation}

Therefore
\begin{equation}
	\begin{aligned}
		H_{\infty}(\pi_{L_{P}}, C^{-1})
		 &= - \log{V(\pi_{L_{P}}, C^{-1})} = - \log1= 0
	\end{aligned}
\end{equation}

Thus the min-leakage is:
\begin{equation}
	\begin{aligned}
	L(\pi_{L_P}, C^{-1}) 
	 &= H_{\infty}(\pi_{L_P}) - H_{\infty}(\pi_{L_{P}}, C^{-1}) \\
	 &= H_{\infty}(\pi_{L_P}) - 0 \\
	 &= H_{\infty}(\pi_{L_P})
	\end{aligned}
\end{equation}

And finally:
\begin{equation}
	ML(C^{-1}) = \sup_{\pi_{L_P}}{L(\pi_{L_P},C^{-1})} =  \sup_{\pi_{L_P}} H_{\infty}(\pi_{L_P}) = \log{|L_P|}
\end{equation}

This result consists with our intuition and the Capacity in \Cref{Eq: Cap in length} that all bits of $l_P$ are leaked through $l_C$.

\subsubsection{Explicit Observables with noncoresec}

As explained in \Cref{Subsubsec: Explicit noncoresec}, the only explicit observables when using noncoresec is the MAC Header. 

Referring to \Cref{Subsec: 802.15.4 MAC}, there are four types of MAC Frames. However, both Beacon and Command frame are solely used for connection maintenance and are independent to the application running on Sensor Nodes; therefore we consider them as non leakage sources as described above.

For the Data frame we explained in \Cref{Subsec: 802.15.4 MAC},
\begin{description}[style=nextline]
	\item[Frame Control]
	Among the flags in Frame Control defined by \cite{802154}, we found only two flags potentially related to upper layer application:
		\begin{description}
			\item[Security Enabled]
			When noncoresec is used, this field is constantly $1$. Due to \Cref{Cor: Constant Leakage}, this is not a potential leakage source.
			\item[ACK Required]
			Since MAC ACK is optional and this field can be set by an upper layer application, we suspect this flag as a potential leakage source.
		\end{description}
	\item[Sequence Number]
	Sequence Number is solely managed by MAC Layer and hence likely to be independent to upper layer application.
	\item[Address Information]
	As explained above, address information is excluded from the potential leakage sources.
	\item[Data]
	Since the encryption is considered to be a random function, we consider the value Data to be independent from Application Data and Cryptographic Key.
	\item[Frame Checksum]
	Since Frame Checksum is computed from the frame itself and therefore contains only redundant information. We argue that Frame Checksum does not induce any additional information leakage to a packet.
\end{description}

For the Auxiliary Security header we explained in \Cref{Subsec: 802154 Sec},

\begin{description}[style=nextline]
	\item[Security Level]
	As we explained in \Cref{Chp: Experiment Setup}, this value is constantly $7$ in our experiments and is not a potential leakage source due to \Cref{Cor: Constant Leakage}.
	\item[Frame Counter]
	Similar to Sequence Number, this field is likely to be independent to application.
	\item[Key Strategy]
	noncoresec does not utilise this field at all and thus is constantly $0$. It is not a leakage source due to \Cref{Cor: Constant Leakage}.
\end{description}

To summarise, among the explicit observables with noncoresec,

\begin{itemize}
	\item ACK Request flag in Frame Control is a sceptical leakage source.
	\item Sequence Number and Frame Counter are likely to be independent to application.
\end{itemize}

\subsubsection{Explicit Observables with DTLS}

\Cref{Subsubsec: Explicit DTLS}

\subsubsection{Leakage of Topology}

\subsection{Conclusion}