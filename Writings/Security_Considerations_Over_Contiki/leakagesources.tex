\chapter{Potential Leakage Sources}

In this section, we discuss the potential leakage sources in the experimental environment we have set up in \Cref{Chp: Experiment Setup}. 

%\section{A Review of Review} \label{Sec: RoR}
%
%Before discussing the leakage, a much more fundamental question to ask is: 
%
%\begin{center}
%What does the term \textit{information} mean?\footnote{
%I am making jokes of myself here...? Although I'd argue that the hardness of this problem can be reduced to ''What is Love?''... I am so going to remove this when I submit this report as a formal document.
%}
%\end{center}
%
%Aside of the obscure mathematical definitions given by Information Theoretic literatures, this turns out to be an ultimately tricky question that cannot be casually answered.  Apparently, the term ``information'' has a huge varieties in literatures we have reviewed in \Cref{Chp: Literature Review}:
%
%\begin{itemize}
%	\item In literatures about the flaws in implementations of algorithms and protocols, ``information'' is the plaintext, usually represented as a numerical value, such as \cite{802154sec} \cite{rfc7457} \cite{CompressionRatioAttack} \cite{PaddingOracle}. Or it could be the secret key materials in many cryptographic side channel attack literatures, such as \cite{DPA}.
%	\item In Traffic Analysis Attacks against web applications, ``informaiton'' is the user input, such as \cite{PinpointWeb} \cite{SearchAttack}, or the content of websites such as \cite{WebSideChannel}.
%	\item When we talk about website fingerprinting such as \cite{WebsiteFingerprint} \cite{Peekaboo} and \cite{PClassifier}, ``information'' is the identity of websites. 
%	\item The category only gets even more messy and trivial when we look at other Traffic Analysis Attacks, such as the spoken language and phrases in \cite{VoIPLanguage} and \cite{VoIPPhrases}, user event and OS of Apple product in \cite{AppleMessage} or even the ambient change and motion event in \cite{Video}.
%\end{itemize}
%
%Even in the Information Theoretic way, some literatures, such as \cite{OneTryGuess} \cite{GLeakage} and \cite{AddMulGLeakage},  people define ``leakage'' based on implicit definitions of ``information''.
%
%Therefore, arguably, to the best of our intellectual\footnote{Sorry, I mean mine...}, the most ``formal'' and ``generalised'' definition we can provide in this project is:
%
%\begin{definition} \label{Def: Information}
%``Information'' is any data that is unknown to the adversary.
%\end{definition}
%
%Linguistically, ``secret'' would be a better word that suites the statement. However, we stick on this definition anyway in this chapter, in a casuistry way.
%
%\section{Definitions about Adversary and Leakage}
%
%Although \Cref{Def: Information} makes sense at its first glance, the logic suddenly becomes fuzzy when we try to introduce ``leakage'', as it immediately will be defined as:
%
%\begin{definition}
%``Leakage'' is any information known to the adversary that the adversary does not know.\footnote{
%I know I will have to rewrite this part anyway but I just don't have any idea how at the moment...
%}
%\end{definition}
%
%To avoid suffocation in the swamp of philosophy, we propose an approach of defining terms in a  ``complementary'' way of \Cref{Def: Information}.
%
%\begin{definition}
%Observable is a data set that is given to and / or generated by the adversary.
%\end{definition}
%
%靠靠靠靠靠。。。我编不下去了

\section{Observables}

We start by defining the data that could potentially be exploited by the adversary based on our setup described in \Cref{Chp: Experiment Setup}.

As we have explained in \Cref{Chp: Experiment Setup}, there are two security measures we consider in this project, namely noncoresec and DTLS. Referring to \Cref{Chp: Building Blocks}, these measures are implemented at different layers. The observables are therefore different for noncoresec and DTLS. 

\subsection{noncoresec}

\subsection{DTLS}

\section{Leakages}

%\begin{itemize}
%	\item In literatures about the flaws in implementations of algorithms and protocols, ``information'' is the plaintext, usually represented as a numerical value, such as \cite{802154sec} \cite{rfc7457} \cite{CompressionRatioAttack} \cite{PaddingOracle}. Or it could be the secret key materials in many cryptographic side channel attack literatures, such as \cite{DPA}.
%	\item In Traffic Analysis Attacks against web applications, ``informaiton'' is the user input, such as \cite{PinpointWeb} \cite{SearchAttack}, or the content of websites such as \cite{WebSideChannel}.
%	\item When we talk about website fingerprinting such as \cite{WebsiteFingerprint} \cite{Peekaboo} and \cite{PClassifier}, ``information'' is the identity of websites. 
%	\item The category only gets even more messy and trivial when we look at other Traffic Analysis Attacks, such as the spoken language and phrases in \cite{VoIPLanguage} and \cite{VoIPPhrases}, user event and OS of Apple product in \cite{AppleMessage} or even the ambient change and motion event in \cite{Video}.
%\end{itemize}

\subsection{``Traces'' in WSN}

\subsection{RSSI}

\section{A Study by Cross Referencing}



