\chapter{Potential Leakage Sources}

In this chapter, we discuss the potential leakage sources in the experimental environment we have set up in \Cref{Chp: Experiment Setup}. 

\section{A Review of Review}
%What is ``Information''

The first and fundamental question that must be clarified in this project is: what is ``information''?

If we look at those attacks described in \Cref{Chp: Literature Review}:

\begin{itemize}
	\item In literatures about the flaws in implementations of algorithms and protocols, ``information'' is the plaintext, usually represented as a numerical value, such as \cite{802154sec} \cite{rfc7457} \cite{CompressionRatioAttack} \cite{PaddingOracle}. Or it could be the secret key materials in many cryptographic side channel attack literatures, such as \cite{DPA}.
	\item In Traffic Analysis Attacks against web applications, ``informaiton'' is the user input, such as \cite{PinpointWeb} \cite{SearchAttack}, or the content of websites such as \cite{WebSideChannel}.
	\item When we talk about website fingerprinting such as \cite{WebsiteFingerprint} \cite{Peekaboo} and \cite{PClassifier}, ``information'' is the identity of websites. 
	\item The category only gets even more messy and trivial when we look at other Traffic Analysis Attacks, such as the spoken language and phrases in \cite{VoIPLanguage} and \cite{VoIPPhrases}, user event and OS of Apple product in \cite{AppleMessage} or even the ambient change and motion event in \cite{Video}.
\end{itemize}

``Information'' is always a relative concept to ``application'', and so is ``leakage''. 

In this report, we experiment with the applications we described in \Cref{Sec: Applications}. Specifically we focus on information leakages through the packet features with regard to the following subjects:

\begin{itemize}
	%Value of Application Data
	%Length of Application Data
	%Application Secret
	%Topology
	\item Content of Application Data
	\item Length of Application Data
	\item Cryptographic Key
	\item Application Code Routine
	\item Topology
\end{itemize}

The Application Data refers to:
\begin{itemize}
	\item CoAP Payload as in \Cref{Subsec: CoAP} when CoAP is used.
	\item DTLS Payload as in \Cref{Subsec: DTLS} when DTLS is used.
	\item UDP Payload as in \Cref{Subsec: UDP} otherwise.
\end{itemize}

%\section{Definitions of ``Information''}
%
%Although ``informaion'' remains an ambiguous term in the context, in this section, we specify ``information'' as practical objects in this project.
%
%\begin{definition}
%	An Application is the code running on a Sensor Node that is not part of Contiki source code.
%\end{definition}
%
%\begin{definition}
%	Application Data refers to any data contained in a packet such that:
%	\begin{itemize}
%		\item If CoAP is used, it is CoAP Payload we explained in \Cref{Subsec: CoAP}.
%		\item Or if DTLS is used, it is DTLS Payload we explained in \Cref{Subsec: DTLS}.
%		\item Otherwise it is the UDP Payload we explained in \Cref{Subsec: UDP}.
%	\end{itemize}
%\end{definition}
%
%\begin{definition}
%	Application Key is the key material stored in a Sensor Node. Specifically,
%	\begin{itemize}
%		\item If noncoresec is used, it is the network shared key.
%		\item If DLTS is used, it is the key used by DTLS.
%	\end{itemize}
%\end{definition}

%Value of Application Data
%Length of Application Data
%Application Secret
%Topology

\section{Packet Features}

%We have explained in \Cref{Chp: Experiment Setup} that there are two security measures implemented on our platform, namely noncoresec and DTLS. Referring to \Cref{Chp: Building Blocks}, these measures are implemented at different layers. The observables are therefore different with noncoresec and DTLS. 
%
%\begin{description}[style=nextline]
%	\item[Observables with noncoresec]
%	\item[Observables with DTLS] 
%\end{description}
%
%
%What is traces in WSN?

\section{What could leak?}






