\chapter{Potential Leakage Sources}

In this chapter, we discuss the potential leakage sources in the experimental environment we have set up in \Cref{Chp: Experiment Setup}. 

\section{A Review of Review} \label{Sec: A Review of Review}
%What is ``Information''

The first and fundamental question that must be clarified in this project is: what is ``information''?

If we look at those attacks described in \Cref{Chp: Literature Review}:

\begin{itemize}
	\item In literatures about the flaws in implementations of algorithms and protocols, ``information'' is the plaintext, usually represented as a numerical value, such as \cite{802154sec} \cite{rfc7457} \cite{CompressionRatioAttack} \cite{PaddingOracle}. Or it could be the secret key materials in many cryptographic side channel attack literatures, such as \cite{DPA}.
	\item In Traffic Analysis Attacks against web applications, ``informaiton'' is the user input, such as \cite{PinpointWeb} \cite{SearchAttack}, or the content of websites such as \cite{WebSideChannel}.
	\item When we talk about website fingerprinting such as \cite{WebsiteFingerprint} \cite{Peekaboo} and \cite{PClassifier}, ``information'' is the identity of websites. 
	\item The category only gets even more messy and trivial when we look at other Traffic Analysis Attacks, such as the spoken language and phrases in \cite{VoIPLanguage} and \cite{VoIPPhrases}, user event and OS of Apple product in \cite{AppleMessage} or even the ambient change and motion event in \cite{Video}.
\end{itemize}

``Information'' is always a relative concept to ``application'', so is ``leakage''. 

In this report, we are interested on information leakages with regard to the following attributes:

\begin{description}[style=nextline]
	\item[Content and Size of Application Data]
	The implication of Application Data varies on different applications.
	
	\item[Cryptographic Key]
	This represents the corresponding key materials in the security measure.
	
	\item[Application Code Routine]
	This implies the application specific feature.
	
	\item[Network Topology]
	This implies how Sensor Nodes connect to each other.
\end{description}

Notice that the information leakage is highly application specific. In this report, we discuss only the applications described in \Cref{Sec: Applications}. 

\section{Observables} \label{Sec: Observables}

Generally speaking, observables in captured traffic can be classified into two categories:
\begin{enumerate}
	\item Implicit observables. These imply the data an adversary can access without inspecting the content of packets.
	\item Explicit observables. These imply the data given in the unencrypted part of packets.
\end{enumerate}

\subsection{Implicit Observables}

There are mainly two observables we concern in this category:

\begin{description}[style=nextline]
	\item[Packet Size] 
	This refers to the size (or length) of a packet. This value is also explicitly included in the packet headers.
	
	\item[Packet Timing]
	This refers to the time being sent of a packet. 
\end{description}

In addition to Packet Size and Packet Timing, there are some other implicit observables we do not concern in this report:

\begin{description}[style=nextline]
	\item[Presence of Packets]
	Apparently, the presence of a packet implies the existence of a Sensor Node device, which is kind of an information leakage. However, we do not take this into account. We assume the adversary has prior knowledge of the existence of WSN.
	
	\item[RSSI]
	Receive Signal Strength Indicator, RSSI, indicates the RF signal strength from a Sensor Node. Practically speaking, by measuring the RSSI of frames sent from a specific source MAC address, the adversary is capable to reveal the physical location of Sensor Node. In this report we do not take RSSI into account as its physical character is beyond the scope of this project. We assume the adversary has prior knowledge of the geographic information of the Sensor Nodes.
\end{description}

The implicit observables are commonly accessible in all scenarios in our experiments.

\subsection{Explicit Observables}

We have explained in \Cref{Chp: Experiment Setup} that there are two security measures implemented on our platform, namely noncoresec and DTLS. Referring to \Cref{Chp: Building Blocks}, these measures are implemented at different layers. The observables are therefore different for noncoresec and DTLS. 

\subsubsection{noncoresec}

As explained in \Cref{Subsec: 802.15.4 Security Implementation in Contiki}, noncoresec is the 802.15.4 Security implementation on Contiki. When noncoresec is enabled, all captured frames is in the form we explained in \Cref{Subsec: 802.15.4 MAC} and \Cref{Subsec: 802154 Sec}. The MAC Layer Header is the only observable part of a packet.

We stated in \Cref{Subsec: noncoresec in experiment} that we always use highest Security Level of 802.15.4 Security; therefore:
\begin{itemize}
	\item MAC Payload is always authenticated and encrypted in AES-128 with CCM* mode.
	\item Security Level is constantly 0x7.
	\item Key Strategy is constantly 0x0 as noncoresec does not implement any key management.
\end{itemize}

As a result of MAC Payload being authenticated and encrypted, 
\begin{itemize}
	\item The adversary cannot join the 6LoWPAN network as we have explained in \Cref{Subsec: noncoresec in experiment}.
	\item The adversary cannot distinguish an IPv6 packet and an ICMPv6 packet by simply looking the contest as it is encrypted. 
\end{itemize}

The IPv6 packets effectively are the packets with Application Data. ICMPv6 packets in our experiments are basically RPL messages.

\subsubsection{DTLS}

As explained in \Cref{Chp: Building Blocks}, when DTLS is used, the explicit observables are: 

\begin{itemize}
	\item MAC Header as explained in \Cref{Subsec: 802.15.4 MAC}.
	
	\item Compressed IPv6 Header as explained in \Cref{Subsec: 6LoWPAN Adaptation Sub Layer}, \Cref{Subsec: IPv6 Data Packets} and \Cref{Subsec: ICMPv6}.
	
	\item UDP Header as explained in \Cref{Subsec: UDP}.
	
	\item DTLS Header as explained in \Cref{Subsec: DTLS}.
\end{itemize}

Since DTLS is built on Application Layer, therefore it does not impose any setting to the lower layer headers. The only exception is the Protocol Version of DTLS Header since tinydtls supports only DTLS version 1.2, therefore Protocol Version is constantly \{0xfe, 0xfd\}, i.e. \{0xff, 0xff\} - \{0x01, 0x02\}.

\subsection{Traces} \label{Subsec: Traces}

Before discussing the ``traces'' in WSN applications, as a comparison, we review the different definition of ``traces'' in the literatures we reviewed in \Cref{Chp: Literature Review}:

\begin{itemize}
	\item In literatures about Traffic Analysis Attacks against web applications\cite{WebSideChannel}\cite{PinpointWeb}\cite{SearchAttack}, a ``trace'' refers to the continuous\footnote{We would argue that the term ``continuous'' is ambiguous; nevertheless it is the best definition we can thought of.} packets triggered by an user input.
	\item In Website Fingerprinting literatures\cite{WebsiteFingerprint} \cite{HClassifier} \cite{PClassifier} \cite{Peekaboo}, a ``trace'' refers to the continuous packets that triggered by a browser requesting a website.
	\item Other attacks use more specific application dependent definitions of a ``trace''. \cite{VoIPLanguage} and \cite{VoIPPhrases} defined a ``trace'' to be the continuous packets during a VoIP conversation. \cite{Video} defines it to be the continuous packets during a video conversation. \cite{AppleMessage} does not even use the term ``trace'' at all\footnote{Actually it did used once as a verb equivalent to ``track''.} as it only analyses a single packet.
\end{itemize}

Practically, the ``trace'' of an Internet application usually can be isolated by filtering packets belong to the same TCP connection, i.e. TCP packets with the same entities\footnote{An entity is the combination of an IP address and a port}. However, we cannot apply the same method on many WSN applications due to the change of protocol suite. Further more, the WSN applications perform Machine To Machine, M2M, communication which has much less external intervention, makes it harder to group packets into ``traces''.

Another factor needs to be taken into concern is the different characteristics of WSN applications and Internet applications. As a result of resource constrained environment, that WSN application designs tend to be simple and stateless, that is to say the computation on the Sensor Node needs to be simple enough to be performed on such devices and data transmission is minimised. From this aspect, we argue that the simple applications we described in \Cref{Sec: Applications} captured these characteristics of WSN applications.  

We explicitly define a model for our WSN applications.

\begin{definition} \label{Def: Session}
A Session is defined as a series of events where a Sensor Node reports Application Data to a Manager. A Session includes an optional Request sent from Manager to Sensor Node, and a Response containing the Application Data sent from Sensor Node to Manager.
\end{definition}

\begin{definition} \label{Def: Trace}
A Trace is defined as all captured packets within one Session.
\end{definition}

In our applications described in \Cref{Sec: Applications}, there are typically only two packets in a trace. The first one corresponds to the Request and the second corresponds to Response. In case of other applications, e.g. the broadcast keyllsec application in \Cref{Sec: Applications} or a CoAP application using OBSERVE option as we explained in \Cref{Subsec: CoAP}, there might be only one Response packet in a trace without a Request.

As of this project we consider only information leakage within single Session. One might argue that there is potential information leakage when jointly analyse traces from different Sessions, such as the frequency of Requests can ``leak'' itself. We do not take these leakage into concern as it does not make much sense without considering within a higher layer application. Also in our experiments, we make a reasonable assumption based on our code that each Session is independent, as each execution of the code is independent to its last execution.

%Further more, in our applications the content of Request is a constant 3 bytes ASCII string ``GET''. We assume this known by the adversary. We make this assumption to focus on the leakage of the Application Data in Responses. 

\section{Theoretical Analysis}

In this section we provide a theoretical analysis over all observables we explained in \Cref{Sec: Observables}. It is done in two approaches:

\begin{enumerate}
	\item Cross reference the intersection of the observables we explained in \Cref{Sec: Observables} with the known exploitable side channel information we introduced in \Cref{Sec: Traffic Analysis Attacks}
	\item Semantically analyse the relation between Application Data and observables. 
\end{enumerate}

We discuss only the leakage over the attributes we specified in \Cref{Sec: A Review of Review}.

For some of the observables and attributes, we argue that:

\begin{theorem} \label{Te: IR}
For the definitions in \Cref{Subsec: Information Theory}, if the observable $Y$ is independent to the secret $X$, then there is no leakage of $X$ over $Y$.
\end{theorem}

\begin{proof}
	%Independent random variables does not leak.
	Since $X$ and $Y$ are independent, therefore
	\begin{eqnarray*}
		\begin{aligned}
			P(x,y) &= P(x)P(y) \\
			P(x|y) &= P(x)
		\end{aligned}
	\end{eqnarray*}

	For Mutual Information and Capacity, we have:
	\begin{eqnarray*}
		\begin{aligned}
			H(X|Y) 
			&= - \sum_{x \in X} \sum_{y \in Y} P(x,y)\log{P(x|y)} \\
			&= - \sum_{x \in X} \sum_{y \in Y} P(x)P(y)\log{P(x)} \\
			&= \sum_{y \in Y} P(y) (- \sum_{x \in X}P(x)\log{P(x)}) \\
			&= \sum_{y \in Y} P(y) H(X) = H(X) \sum{_y \in Y} P(y) \\
			&= H(X)
		\end{aligned}
	\end{eqnarray*}
	
	Therefore
	\begin{eqnarray*}
		\begin{aligned}
			I(X;Y) &= H(X) - H(X|Y) = H(X) - H(X) = 0 \\
			C &= \sup_{\forall P(X)} I(X;Y) = \sup_{\forall P(X)} 0 = 0
		\end{aligned}
	\end{eqnarray*}
	
	Similarly for gain function based leakage\cite{GLeakage},
	\begin{eqnarray*}
		\begin{aligned}
			V_{g}(\pi, C) 
			&= \sum_{y \in Y}{\max_{w \in W}\sum_{x \in X}{\pi[x]C[x,y]g(w,x)}} \\
			&= \sum_{y \in Y}{\max_{w \in W}\sum_{x \in X}{\pi[x]P(y|x)g(w,x)}} \\
			&= \sum_{y \in Y}{\max_{w \in W}\sum_{x \in X}{\pi[x]P(y)g(w,x)}} \\
			&= \sum_{y \in Y}p(y){\max_{w \in W}\sum_{x \in X}{\pi[x]g(w,x)}} \\
			&= \max_{w \in W}\sum_{x \in X}{\pi[x]g(w,x)} = V_{g}(\pi)
		\end{aligned}
	\end{eqnarray*}
	
	Therefore
	\begin{equation*}
		H_g(\pi, C) = -\log{V_g(\pi, C)} = -\log{V_g(\pi)} = H_g(\pi)
	\end{equation*}
	
	Hence 
	\begin{eqnarray*}
		\begin{aligned}
			L_g(\pi, C) &= H_g(\pi) - H_g(\pi,C) = H_g(\pi) - H_g(\pi) = 0\\
			ML_g(C) &= \sup_{\pi} L_g(\pi, C) = \sup_{\pi} 0 = 0
		\end{aligned}
	\end{eqnarray*}
\end{proof}

\begin{corollary} \label{Cor: Constant Leakage}
	An observable with a constant value has no leakage under the definitions in \Cref{Subsec: Information Theory}.
\end{corollary}

\begin{proof}
	This directly followed by the fact that a fixed observable $y$ with $P(y) = 1$ is independent to the secret $X$.
\end{proof}

\subsection{Cross Reference of Observables with Known Attack} \label{Subsec: Cross Reference}

Despite the differences in the nature applications, we discuss the applicability of the packet features those have been exploited in Traffic Analysis Attacks, as we have summarised in \Cref{Tbl: Classifiers in Traffic Analysis Literatures}.

\begin{description}[style=nextline]
	\item[Direction]
	For a two packets Session, it is predictable that the first Request packet is from Manager to Sensor Node and the second Response packet from Sensor Node to Manager. In a one packet Session there is only one packet from Sensor Node to Manager.
	
	\item[Length]
	The same feature can be observed as an implicit observable.
	
	\item[Frequency Distribution of Length]
	The same feature can be computed by packet length. However, since there are typically only two packets in a trace, the result is likely to be $0.5$ for the length of Request packet and $0.5$ for the length of Response packet. In a one packet Session there is only one value in the distribution with probability of $1$.
	
	\item[Size, HTML and Number Markers]
	In a two packet Session there is only one direction change in a trace; thus the marker are constant marks the second packet. In one packet Session this feature is simply not applicable.
	
	\item[Total Bytes]
	The same feature can be computed through packet lengths.
	
	\item[Percentage Incoming Packets]
	The term ``incoming'' refers to the direction of web server to the browser in its original Web Fingerprint literature. In our experiments we assumed the adversary monitors all packets in the network; thus there is not an explicit definition of ``incoming'' and ``outgoing''. Further more, in our applications, there is only two packets in a Session for each direction, or only packet from Sensor Node to Manager.
	
	\item[Number of Packets]
	As explained in \Cref{Subsec: Traces}, this value is either $1$ or $2$ in our experiments, depends on the application running. 
	
	\item[Total Time]
	In a two packet Session this is exactly the interval between Request packet and Response packet. In a one packet session this is not applicable.
	
	\item[Total Per-direction Bandwidth]
	Since there is at most only one packet at each direction, this feature is effectively a single packet size divided by total time for each direction.
	
	\item[Traffic Burst]
	Traffic burst is reduced packet size in our applications as there is at most only one packet each direction.
\end{description}

As we can see from the description above, several packet features are fixed or not applicable in our applications, including:

\begin{itemize} 
	\item Direction
	\item Frequency Distribution of Length
	\item Size, HTML and Number Markers
	\item Percentage Incoming Packets,
	\item Number of Packets
\end{itemize}

According to \Cref{Cor: Constant Leakage}, we consider these packet features as non leakage features. Although arguably these features may be exploited to leak information of which application is running over the Sensor Nodes, in this report, we assume the adversary has this prior knowledge and hence do not consider such as an information leakage.

As a result, the potential leakage sources from the literatures are:
\begin{itemize}
	\item Length
	\item Total Bytes
	\item Total Time
	\item Total Per-direction Bandwidth
\end{itemize}

Notice that we ignored Traffic Bursts since it is reduced to packet length in our applications as explained above.

\subsection{Semantic Analysis}

In this section, we provide a semantic analysis over the observables based on their semantics we explained in \Cref{Chp: Building Blocks}.

As we have explained in \Cref{Chp: Building Blocks}, the 6LoWPAN protocol is defined layer by layer; therefore the processing of lower layer protocol is usually independent from the upper layer protocols. For example, the MAC Sequence Number is managed solely at MAC Layer and thus we consider it as an independent variable to any observables in upper layers as well as the Application Data. As a counter example, the Next Header field in IPv6 header may depend on the whether TCP or UDP is used at Transport Layer, or specifies an ICMPv6 header.

As we have claimed in \Cref{Te: IR}, we argue that there is no leakage in observable lower layer headers those are independent to upper layer payload.

\subsubsection{Implicit Observables}

The implicit observables are effectively covered by the potential leakage sources we analysed in \Cref{Subsec: Cross Reference}. As a matter of fact, these observables turned out to be the most exploitable features in the literatures we have reviewed in \Cref{Sec: Traffic Analysis Attacks}.

\subsubsection{Explicit Observables with noncoresec}

As we have explained 

\subsubsection{Explicit Observables with DTLS}

\subsubsection{Leakage of Topology}

\subsection{Conclusion}