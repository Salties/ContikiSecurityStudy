\chapter{DTLS} \label{Chp: DTLS}
%DTLS has potentially the best interoperability as it is an variation of the widely used TLS in Internet. However, its design might not fit into the nature of WSN for practical reasons.
%
%\section{Implementation Issues}
%The most practical implementation we found on Contiki OS so far is tinydtls.
%
%tinydtls\cite{tinydtls} currently supports two ciphersuites, namely TLS\_PSK\_WITH\_AES\_128\_CCM\_8 and TLS\_ECDHE\_ECDSA\_WITH\_AES\_128\_CCM\_8. 
%
%However, we encountered several difficulties when trying to set up an encrypted network using tinydtls.
%
%\begin{description}
%\item[Low Computational Power] \hfill \\
%Curve computation requires relatively a large amount of computational power. Even using a relatively powerful platform (CC2538), it still takes minutes to complete a DTLS handshake with
%TLS\_ECDHE\_ECDSA\_WITH\_AES\_128\_CCM\_8.
%
%\item[Low Bandwidth] \hfill \\
%The 6LowPAN standard specifies that the minimum MTU is 127 bytes whilst 67 (87 with LLSEC) bytes are occupied by protocol headers until UDP, which leaves 60 (40 with LLSEC) bytes available for UDP layer payload. This value has been exceeded by several handshake packets even with pre-shared keys. Doing key exchange or even using longer keys only makes this problem worse. Some attempts have been made to solve this issue, e.g. CoDTLS\cite{CoDTLS}\footnote{This draft has been abandoned for some reason we do not know.}. As a result, DTLS is only available on those devices support extra frame length than 6LowPAN requirements.
%
%\item[Code Size] \hfill \\
%The tinydtls fails to fit into some devices, e.g. skymote, as its size of code is too large.
%\end{description}
%
%Therefore although TLS\_PSK\_WITH\_AES\_128\_CCM\_8 is less flexible (and probably less secure) as it uses a pre-shared master secret, it is still considered to be a relatively practical security measure as it requires less resources.
%
%\section{No Multicast Support}
%Some application protocols, such as CoAP, utilises the multicast feature of 6LowPAN whilst TLS is a protocol designated for securing communications between two parties, so is DTLS. To  our knowledge, DTLS does not make any attempt to support multicasting.
%
%\section{Overloading DTLS with LLSEC}
%Adopting both security measures at the same time is possible as they are implemented at different layers. However, it is questionable whether this will bring more security, as both {\it noncoresec} and TLS\_PSK\_WITH\_AES\_128\_CCM\_8 are using 128 bit AES with CCM mode as their cryptographic primitive.
