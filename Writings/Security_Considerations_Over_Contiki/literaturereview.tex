\chapter{Literature Review} \label{Chp: LiteratureReview}

In this chapter we review the literatures with respect to security in WSN. We roughly categorise the literatures into three aspects we consider to be mostly relevant to this project:

\begin{enumerate}
	\item Protocol and implementation flaws
	\item Traffic analysis techniques
	\item Information leakage detection
\end{enumerate}

As introduced in \Cref{Chp: Building Blocks}, many 6LoWPAN protocols are derived from Internet. Therefore in this report we also refer to existing attacks on Internet that could potentially be applied in WSN.

\section{Protocol and Implementation Flaws}

In this section we focus on protocol suite described in \Cref{Tbl: Summary of WSN Building Blocks}, namely `'802.15.4 + 6LoWPAN + CoAP''.

%Security consideration of 802.15.4
\cite{802154sec} reviews the design flaw in 802.15.4 Security. First of all, the nonce reuse may happen due to the same key used in different entry, or power failure on device. Secondly, the key management also needs to be improved. The ACL does not support group key, using a default ACL entry for network shared key is incompatible with replay protection. Finally, the Security Levels without authentication should not be used. Allowing an adversary to forge messages potentially breaches the data confidentiality in many applications. It also allows the adversary to launch a Denial of Service, DoS, attack by spoofing a frame with the maximum Frame Counter which eventually triggers the replay protection on later legitimate frames. In addition, the lack of security in ACK frames can also be exploited in a jamming DoS attack to forge false ACKs to prevent retransmission.

%6LoWPAN Fragmentation Attack
6LoWPAN fragmentation attack\cite{6lpFragAtk} exploits the unauthenticated packet fragments in 6LoWPAN. An adversary within the network can spoof a fragment of IPv6 packet causing the receiver to drop the corrupted  packet, or she can send a forged initial fragment requesting the victim node to allocate unnecessary memory which eventually results into legal packets being dropped. They propose two countermeasures to prevent these fragmentation attacks. The first countermeasure is to add chained integrity tag into each fragment to prevent the spoofed fragments. The second countermeasure is to use a more sophisticated memory management scheme that does not allocate memory until an actual fragment is received.

%6LoWPAN DoS
\cite{6lpRplAtk} gives an overview of attacks that targets 6LoWPAN and RPL. These attacks mainly results into malfunctioning of the WSN. In Sinkhole Attack\cite{Sinkhole}, the malicious node sends false RPL message to direct all messages to itself. Sinkhole Attack can further extend to Blockhole Attack\cite{Blackhole} by dropping all packets silently, eventually disables the communication of network. Wormhole Attack\cite{Wormhole} works by replaying legitimate RPL messages at an illegal location, causing confusion in the DODAG structure and therefore disrupts the communication. A malicious node launching Hello Flood Attack repeatedly broadcasts a Hello, refers to DIS, message, triggers its neighbour to respond with DIO messages and eventually depletes the battery of victims.

%Compression Ratio Attack
\cite{CompressionRationAttack} describes a plaintext recovery attack that exploits the length of compressed ciphertext. This type of attack has been realised by the CRIME\cite{CRIME} attack and BREACH\cite{BREACH} attack that is against TLS compression and HTTP compression respectively. In this type of attacks, the plaintext constitutes of two parts. The first part is the explicit part which is known or even controlled by adversary. The second part is the implicit part which is unknown secret to the adversary . The first observation is that the plaintext is compressed before encrypted, since the encrypted ciphertext should appears random and can be hardly compressed. The second observation is that the more repetitive patterns in the plaintext the more bytes it will be compressed and hence results into more shrink in the length of ciphertext. The two factors combined together gives the adversary an oracle to guess the implicit part of plaintext through the explicit part, as a correct guess in the explicit part corresponds to a higher compression ratio in ciphertext.\cite{CompressionCountermeasure} proposed two countermeasures to this types of attacks. The first one is to compress the explicit part and implicit part separately. This approach completely disables the compression oracle but is not generally applicable as the explicit part and implicit part are application dependent. The second countermeasure is to use a fixed dictionary in compression. This countermeasure also prevent such attacks but suffers in performance.

%Lucky 13

%Smartgrid Dump Crypto

\section{Traffic Analysis Techniques}

\section{Information Leakage Detection}

