\chapter{Introduction}
%Contiki OS is an embedded system that used to build WSN\footnote{Wireless Sensor Network} based on 802.15.4\cite{802154} compatible devices and 6LowPAN\cite{rfc4944}. This paper discusses two security measurements, namely Link Layer Security (LLSEC) and Datagram TLS (DTLS), within Contiki OS. We also discuss some potential methods of fingerprinting an application running on a sensor node.
%
%In \Cref{Chp: LLSEC} we describes a LLSEC implementation in Contiki OS called \textit{noncoresec} and argues that it does not met certain cryptographic security notions.
%
%\Cref{Chp: DTLS} discusses some implementation issues of DTLS on Contiki OS.
%
%\Cref{Chp: Appdetect} first argues that under some natures of WSN, an adversary could possibly collect more accurate timing information than usual Internet Web-application attacker. Then we describe a potential side-channel attack that fingerprints an application the target sensor node is running, using interactions between PING protocol and the application running on the target node.
Recent technology advance has drawn both industrial and academical attention  to the Internet of Things, IoT. Comparing to traditional embedded devices, the latest hardware are more compacted while possessing incomparable computational power. These advantages enabled the devices to perform more computation at the field and thus inspires us to built more intelligent and robust applications, such as smart houses and smart cities. 