\chapter{Link Layer Security}

Link Layer Security, or LLSEC, is a security measure that implements cryptography just above the physical layer.

Introducing cryptography at a lower level has several benefits. Firstly, more data being encrypted reduces the observable packet features to an adversary, such as SRC\footnote{Source Address} and DST\footnote{Destination Address} field in the IP header which are very likely to be exploited by the adversary. Secondly, authentication at lower level also prevents an active adversary from joining the network which therefore weakens his power. 

Imposing cryptography at a lower level also brings more challenge to the design of sensor network architecture. The first problem is its overhead. Even for a node that only tries to retransmits the packet to its next hop, it must decrypt the whole packet to extract its routing information, and then re-encrypt it before retransmission. This is particularly problematic in a mesh wireless sensor network as it could potentially lead to performance and energy consumption problems. Key management is also challenging due to the lossy and power optimised nature of wireless sensor network.

It is also noticeable that some packet features are not hidden even with LLSEC enabled, such as packet length, timing information and part of the MAC header.

\section{Non core security} \label{sec: noncoresec}

\section{802.15.4 security}

\section{Reset Problem}
\subsection{Initial Vector}

\section{Distinctive packet length for RPL packets}
