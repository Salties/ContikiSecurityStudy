\section{Introduction}
Applications for the Internet of Things (IoT) flourish, leaving a great desire for not only energy efficient, cheap devices, but also for devices that support basic cryptographic functionality such as confidentiality and/or authenticity. Popular algorithms are e.g. the Advanced Encryption Standard\cite{AES} (AES) for confidentiality, and the Elliptic Curve Digital Signature Algorithm\cite{ECDSA} (ECDSA) for authenticity, which, when used in conjunction, enable applications to establish secure end-to-end channels via e.g. Datagram TLS\cite{DTLS} (DTLS). 

However whilst AES is a secure block cipher, one might require randomness to turn it into a secure encryption scheme for arbitrary length messages. Somewhat similarly, ECDSA relies on a known-to-be-secure mathematical problem. However, it also requires large and securely generated random numbers. Consequently, when supporting cryptography the secure generation of random numbers is crucial. 

%The prospering IoT applications constantly proposes higher requirements to security where cryptography plays an important role. Among all prerequisite of implementing cryptographic primitives on these IoT devices, a reliable Random Number Generator (RNG) is critical as it is required in most cryptographic algorithms. In practice, a RNG typically involves a Pseudo Random Number Generator (PRNG) seeded by a high entropy physical source.

In 2013, Texas Instruments (TI) launched a new System-on-chip (SoC), the CC2538\cite{CC2538}, featuring secure channels over 802.15.4 via multiple cryptographic hardware accelerators. Partially because these cryptographic accelerators, projects such as Contiki\cite{Contiki} and OpenWSN\cite{OpenWSN} began to support the CC2538 with enthusiasm. As of writing this paper, the chip features in the suggested list for Zigbee and 6LoWPAN solutions on TI's website\cite{ZigbeeProducts}.

However, despite all the cryptographic hardware support, the CC2538 does not have a Random Number Generator (RNG) dedicated for cryptographic applications; instead, the user manual suggests to use a 16 bit Linear Feedback Shift Register (LFSR) as a Pseudo RNG (PRNG) where the seed is generated by the Radio Frequency (RF) module sampling from the radio noise. Whilst the user guide at no point suggested that this method should be used in conjunction with cryptographic algorithms, developers have little choice in the absence of alternatives. Also, in the absence of published attacks, there is often a temptation to ignore warnings towards insecure RNG implementations such as in \cite{SmartMeterBlog}. 

\subsection{Our Contribution}
We show in this study that this choice proves catastrophic for cryptographic applications, not only because the in-built PRNG has only 16 bit entropy which can be easily predicted, but also because we are able to practically demonstrate how to use radio jamming to bias the seed obtained from the RF module. Consequently, even if the weak in-built PRNG was replaced by a stronger component, the source for the seed could still be tampered with and thus render the system insecure. All the experimental work in this paper are performed on the latest Contiki release version 3.0.
%The related source code can be accessed at \cite{prngtest}.

Our paper is structured as follows. We begin in \Cref{ContikiDriverIssue} with some Contiki RNG driver issues for CC2538. \Cref{LFSR} revises why using a 16 bit LFSR as PRNG is a bad practice and we show how this design flaw can be exploited to break DTLS in \Cref{BreakDTLS}, before reviewing the problem in \Cref{PRNGReflection}. In \Cref{Seed} we explain how CC2538 samples the radio noise into random seed and then we demonstrate how it can be biased by jamming signals in \Cref{Jamming}. Finally we conclude the paper in \Cref{Conclusion}.

\subsection{Related Work}
The design flaw of using a 16 bit LFSR as PRNG has been reported by \cite{SmartMeterBlog}\cite{CC2530PRNG} on CC2430\cite{CC2430Manual} and CC2530\cite{CC2530Manual} respectively. These chips are the predecessors of CC2538 in the SimpleLink\texttrademark  series and they all adopted the same RNG design. The blogs reported the flaw and warned that it could easily be exploited to compromise the Z-Stack library\cite{ZStack} and Smart Energy Profile ECC in many Smart Meter applications. We essentially `rediscovered' that this poor design choice still features in the CC2538 product. However, whilst in previous work the possibility of injecting a jamming signal was contemplated, we are the first to actually examine the technical feasibility of this and to demonstrate a working attack.


\subsection{Contiki Driver Issues}\label{ContikiDriverIssue}
We made extensive use of Contiki in our research and fixed (and reported) some coding issues in the CC2538 RNG driver (Contiki release-3.0). These were, the reading out of the LFSR without ready check, a lack of validity check when reading random seed bits from the RF module, and a bug that drops the Most Significant Bit (MSB) and leaves the Least Significant Bit (LSB) to be constantly zero in the seed.  We modified the code and fixed these issues in our experiments. 

Another issue in the driver is that the CC2538 User's Guide\cite{CC2538Manual} suggests only to use the lower byte (8 bits) as a random number but the driver actually used 16 bits in the LFSR. However, this coding mistake does not affect our result, as will be explained in  later sections.