\chapter{Toys}

In this hello-world set-up, there is only one server and one client connected through local-link. The protocol suit we adopted is: [IPv4 or IPv6] + UDP + DTLS.  
\\
\textbf{Abbreviations}:
\begin{description}
\item[CLIENT] Client.
\item[SERVER] Server.
\end{description}



\section{Odd or Even}
\textbf{Odd or Even} is an extremely simple toy application. It is designed to demonstrate the fundamental idea of traffic analysis.

\subsection{Description}
CLIENT randomly generates a 32-bit unsigned integer R and sends it to SERVER. SERVER replies with a string \lq\lq{}ODD\rq\rq{} or \lq\lq{}EVEN\rq\rq{} according to the integer sent.\ref{Fig: Odd or Even}

\begin{figure}[H] \label{Fig: Odd or Even}
\caption{Description of an Odd-or-Even session}
\centering
\resizebox{8cm}{!}
{% Graphic for TeX using PGF
% Title: /home/yy12135/MyGit/tinyDTLS-Traffic-Analysis/Writings/First-year-review_20150422/Pics/OddOrEven.dia
% Creator: Dia v0.97.2
% CreationDate: Mon Mar  2 17:12:59 2015
% For: yy12135
% \usepackage{tikz}
% The following commands are not supported in PSTricks at present
% We define them conditionally, so when they are implemented,
% this pgf file will use them.
\ifx\du\undefined
  \newlength{\du}
\fi
\setlength{\du}{15\unitlength}
\begin{tikzpicture}
\pgftransformxscale{1.000000}
\pgftransformyscale{-1.000000}
\definecolor{dialinecolor}{rgb}{0.000000, 0.000000, 0.000000}
\pgfsetstrokecolor{dialinecolor}
\definecolor{dialinecolor}{rgb}{1.000000, 1.000000, 1.000000}
\pgfsetfillcolor{dialinecolor}
% setfont left to latex
\definecolor{dialinecolor}{rgb}{0.000000, 0.000000, 0.000000}
\pgfsetstrokecolor{dialinecolor}
\node[anchor=west] at (11.000000\du,9.000000\du){};
\definecolor{dialinecolor}{rgb}{1.000000, 1.000000, 1.000000}
\pgfsetfillcolor{dialinecolor}
\pgfpathellipse{\pgfpoint{31.003315\du}{10.006491\du}}{\pgfpoint{1.831366\du}{0\du}}{\pgfpoint{0\du}{1.837164\du}}
\pgfusepath{fill}
\pgfsetlinewidth{0.100000\du}
\pgfsetdash{}{0pt}
\pgfsetdash{}{0pt}
\pgfsetmiterjoin
\definecolor{dialinecolor}{rgb}{0.000000, 0.000000, 0.000000}
\pgfsetstrokecolor{dialinecolor}
\pgfpathellipse{\pgfpoint{31.003315\du}{10.006491\du}}{\pgfpoint{1.831366\du}{0\du}}{\pgfpoint{0\du}{1.837164\du}}
\pgfusepath{stroke}
% setfont left to latex
\definecolor{dialinecolor}{rgb}{0.000000, 0.000000, 0.000000}
\pgfsetstrokecolor{dialinecolor}
\node at (31.003315\du,10.201491\du){SERVER};
\pgfsetlinewidth{0.100000\du}
\pgfsetdash{}{0pt}
\pgfsetdash{}{0pt}
\pgfsetbuttcap
{
\definecolor{dialinecolor}{rgb}{0.000000, 0.000000, 0.000000}
\pgfsetfillcolor{dialinecolor}
% was here!!!
\pgfsetarrowsend{stealth}
\definecolor{dialinecolor}{rgb}{0.000000, 0.000000, 0.000000}
\pgfsetstrokecolor{dialinecolor}
\draw (31.002868\du,11.893189\du)--(31.000000\du,24.000000\du);
}
\pgfsetlinewidth{0.100000\du}
\pgfsetdash{}{0pt}
\pgfsetdash{}{0pt}
\pgfsetbuttcap
{
\definecolor{dialinecolor}{rgb}{0.000000, 0.000000, 0.000000}
\pgfsetfillcolor{dialinecolor}
% was here!!!
\pgfsetarrowsend{to}
\definecolor{dialinecolor}{rgb}{0.000000, 0.000000, 0.000000}
\pgfsetstrokecolor{dialinecolor}
\draw (15.036382\du,13.528274\du)--(31.002048\du,15.352278\du);
}
\pgfsetlinewidth{0.100000\du}
\pgfsetdash{}{0pt}
\pgfsetdash{}{0pt}
\pgfsetbuttcap
{
\definecolor{dialinecolor}{rgb}{0.000000, 0.000000, 0.000000}
\pgfsetfillcolor{dialinecolor}
% was here!!!
\pgfsetarrowsend{to}
\definecolor{dialinecolor}{rgb}{0.000000, 0.000000, 0.000000}
\pgfsetstrokecolor{dialinecolor}
\draw (31.001229\du,18.811367\du)--(15.012127\du,20.509425\du);
}
\definecolor{dialinecolor}{rgb}{1.000000, 1.000000, 1.000000}
\pgfsetfillcolor{dialinecolor}
\pgfpathellipse{\pgfpoint{15.048531\du}{10.031365\du}}{\pgfpoint{1.729717\du}{0\du}}{\pgfpoint{0\du}{1.701399\du}}
\pgfusepath{fill}
\pgfsetlinewidth{0.100000\du}
\pgfsetdash{}{0pt}
\pgfsetdash{}{0pt}
\pgfsetmiterjoin
\definecolor{dialinecolor}{rgb}{0.000000, 0.000000, 0.000000}
\pgfsetstrokecolor{dialinecolor}
\pgfpathellipse{\pgfpoint{15.048531\du}{10.031365\du}}{\pgfpoint{1.729717\du}{0\du}}{\pgfpoint{0\du}{1.701399\du}}
\pgfusepath{stroke}
% setfont left to latex
\definecolor{dialinecolor}{rgb}{0.000000, 0.000000, 0.000000}
\pgfsetstrokecolor{dialinecolor}
\node at (15.048531\du,10.226365\du){CLIENT};
\pgfsetlinewidth{0.100000\du}
\pgfsetdash{}{0pt}
\pgfsetdash{}{0pt}
\pgfsetbuttcap
{
\definecolor{dialinecolor}{rgb}{0.000000, 0.000000, 0.000000}
\pgfsetfillcolor{dialinecolor}
% was here!!!
\pgfsetarrowsend{stealth}
\definecolor{dialinecolor}{rgb}{0.000000, 0.000000, 0.000000}
\pgfsetstrokecolor{dialinecolor}
\draw (15.042445\du,11.782986\du)--(15.000000\du,24.000000\du);
}
\pgfsetlinewidth{0.100000\du}
\pgfsetdash{}{0pt}
\pgfsetdash{}{0pt}
\pgfsetbuttcap
\pgfsetmiterjoin
\pgfsetlinewidth{0.100000\du}
\pgfsetbuttcap
\pgfsetmiterjoin
\pgfsetdash{}{0pt}
\definecolor{dialinecolor}{rgb}{1.000000, 1.000000, 1.000000}
\pgfsetfillcolor{dialinecolor}
\pgfpathmoveto{\pgfpoint{20.833333\du}{13.000000\du}}
\pgfpathlineto{\pgfpoint{24.166667\du}{13.000000\du}}
\pgfpathcurveto{\pgfpoint{24.626904\du}{13.000000\du}}{\pgfpoint{25.000000\du}{13.447715\du}}{\pgfpoint{25.000000\du}{14.000000\du}}
\pgfpathcurveto{\pgfpoint{25.000000\du}{14.552285\du}}{\pgfpoint{24.626904\du}{15.000000\du}}{\pgfpoint{24.166667\du}{15.000000\du}}
\pgfpathlineto{\pgfpoint{20.833333\du}{15.000000\du}}
\pgfpathcurveto{\pgfpoint{20.373096\du}{15.000000\du}}{\pgfpoint{20.000000\du}{14.552285\du}}{\pgfpoint{20.000000\du}{14.000000\du}}
\pgfpathcurveto{\pgfpoint{20.000000\du}{13.447715\du}}{\pgfpoint{20.373096\du}{13.000000\du}}{\pgfpoint{20.833333\du}{13.000000\du}}
\pgfusepath{fill}
\definecolor{dialinecolor}{rgb}{0.000000, 0.000000, 0.000000}
\pgfsetstrokecolor{dialinecolor}
\pgfpathmoveto{\pgfpoint{20.833333\du}{13.000000\du}}
\pgfpathlineto{\pgfpoint{24.166667\du}{13.000000\du}}
\pgfpathcurveto{\pgfpoint{24.626904\du}{13.000000\du}}{\pgfpoint{25.000000\du}{13.447715\du}}{\pgfpoint{25.000000\du}{14.000000\du}}
\pgfpathcurveto{\pgfpoint{25.000000\du}{14.552285\du}}{\pgfpoint{24.626904\du}{15.000000\du}}{\pgfpoint{24.166667\du}{15.000000\du}}
\pgfpathlineto{\pgfpoint{20.833333\du}{15.000000\du}}
\pgfpathcurveto{\pgfpoint{20.373096\du}{15.000000\du}}{\pgfpoint{20.000000\du}{14.552285\du}}{\pgfpoint{20.000000\du}{14.000000\du}}
\pgfpathcurveto{\pgfpoint{20.000000\du}{13.447715\du}}{\pgfpoint{20.373096\du}{13.000000\du}}{\pgfpoint{20.833333\du}{13.000000\du}}
\pgfusepath{stroke}
% setfont left to latex
\definecolor{dialinecolor}{rgb}{0.000000, 0.000000, 0.000000}
\pgfsetstrokecolor{dialinecolor}
\node at (22.500000\du,14.200000\du){32bit R};
\pgfsetlinewidth{0.100000\du}
\pgfsetdash{}{0pt}
\pgfsetdash{}{0pt}
\pgfsetbuttcap
\pgfsetmiterjoin
\pgfsetlinewidth{0.100000\du}
\pgfsetbuttcap
\pgfsetmiterjoin
\pgfsetdash{}{0pt}
\definecolor{dialinecolor}{rgb}{1.000000, 1.000000, 1.000000}
\pgfsetfillcolor{dialinecolor}
\pgfpathmoveto{\pgfpoint{20.228125\du}{18.000000\du}}
\pgfpathlineto{\pgfpoint{25.140625\du}{18.000000\du}}
\pgfpathcurveto{\pgfpoint{25.818900\du}{18.000000\du}}{\pgfpoint{26.368750\du}{18.447715\du}}{\pgfpoint{26.368750\du}{19.000000\du}}
\pgfpathcurveto{\pgfpoint{26.368750\du}{19.552285\du}}{\pgfpoint{25.818900\du}{20.000000\du}}{\pgfpoint{25.140625\du}{20.000000\du}}
\pgfpathlineto{\pgfpoint{20.228125\du}{20.000000\du}}
\pgfpathcurveto{\pgfpoint{19.549850\du}{20.000000\du}}{\pgfpoint{19.000000\du}{19.552285\du}}{\pgfpoint{19.000000\du}{19.000000\du}}
\pgfpathcurveto{\pgfpoint{19.000000\du}{18.447715\du}}{\pgfpoint{19.549850\du}{18.000000\du}}{\pgfpoint{20.228125\du}{18.000000\du}}
\pgfusepath{fill}
\definecolor{dialinecolor}{rgb}{0.000000, 0.000000, 0.000000}
\pgfsetstrokecolor{dialinecolor}
\pgfpathmoveto{\pgfpoint{20.228125\du}{18.000000\du}}
\pgfpathlineto{\pgfpoint{25.140625\du}{18.000000\du}}
\pgfpathcurveto{\pgfpoint{25.818900\du}{18.000000\du}}{\pgfpoint{26.368750\du}{18.447715\du}}{\pgfpoint{26.368750\du}{19.000000\du}}
\pgfpathcurveto{\pgfpoint{26.368750\du}{19.552285\du}}{\pgfpoint{25.818900\du}{20.000000\du}}{\pgfpoint{25.140625\du}{20.000000\du}}
\pgfpathlineto{\pgfpoint{20.228125\du}{20.000000\du}}
\pgfpathcurveto{\pgfpoint{19.549850\du}{20.000000\du}}{\pgfpoint{19.000000\du}{19.552285\du}}{\pgfpoint{19.000000\du}{19.000000\du}}
\pgfpathcurveto{\pgfpoint{19.000000\du}{18.447715\du}}{\pgfpoint{19.549850\du}{18.000000\du}}{\pgfpoint{20.228125\du}{18.000000\du}}
\pgfusepath{stroke}
% setfont left to latex
\definecolor{dialinecolor}{rgb}{0.000000, 0.000000, 0.000000}
\pgfsetstrokecolor{dialinecolor}
\node at (22.684375\du,19.200000\du){"ODD"/"EVEN"};
\end{tikzpicture}
}
\end{figure}

\subsection{Analysis [To be completed...]}

For every Odd-or-Even session, 

Packets from CLIENT to SERVER:

All fields for every packet are the same, except:
1. Encrypted Application Data field in DTLS layer.
2. Sequence Number increased by 1 every packet.
3. Checksum in UDP layer.

Packets from SERVER to CLIENT:

All fields are the same for every packet except:
1. Encrypted Application Data field in DTLS layer.
2. Sequence Number increased by 1 every packet.
3. Checksum in UDP layer.
4. Length field in both DTLS layer and UDP layer. The values are always (20,41) respectively when data is "Odd" and (21,42) when data is "Even".

Therefore in this application, given pre-knowledge that server responds with either "Odd" or "Even", the length field in both DTLS layer and UDP layer can directly leak the plaintext. 

\section{Leaky Coffee}

\subsection{Description}
\textbf{Leaky Coffee} simulates the scenario that CLIENT initiates a Leaky-Coffee session with a request to SERVER, SERVER replies with a response and CLIENT then reacts according to the response.

\subsubsection{Syntax}
\begin{definition}
\textit{COFFEE} is a set of strings defined as:\\
 $COFFEE = \{  {\text{"AMERICANO"}}, \text{"CAPPUCCINO"}, \text{"ESPRESSO"}, \text{"MOCHA"}\}$
\end{definition}

\begin{definition}
Let '*' represents SUGAR and '@' represents MILK respectively, we denote $n_*$ and $n_@$ as the number of appearances of '*' and '@' in a string. We also call $n_*$ and $n_@$ the degree of SUGAR and MILK of a string.
\end{definition}

\begin{definition}
We define a set of string \textit{ADDITIVE} as:\\
$ADDITIVE = \{\{ SUGAR, MILK \}^{0 - 6} | 0 \leq n_{*} \leq 3, 0 \leq n_{@} \leq 3 \}$.

In another word, an instance of \textit{ADDITIVE} contains no more than 3 SUGAR and MILK.
\end{definition}

\subsubsection{Leaky-Coffee Session}
A Leaky-Coffee session can be described as\ref{Fig: Leaky-Coffee Session}:

\begin{figure}[H] \label{Fig: Leaky-Coffee Session}
\caption{Description of a Leaky-Coffee session}
\centering
\resizebox{14cm}{!}
{% Graphic for TeX using PGF
% Title: /home/yy12135/Google Drive/Writings/First-year-review_20150422/Pics/LeakyCoffee.dia
% Creator: Dia v0.97.2
% CreationDate: Tue Mar  3 14:05:39 2015
% For: yy12135
% \usepackage{tikz}
% The following commands are not supported in PSTricks at present
% We define them conditionally, so when they are implemented,
% this pgf file will use them.
\ifx\du\undefined
  \newlength{\du}
\fi
\setlength{\du}{15\unitlength}
\begin{tikzpicture}
\pgftransformxscale{1.000000}
\pgftransformyscale{-1.000000}
\definecolor{dialinecolor}{rgb}{0.000000, 0.000000, 0.000000}
\pgfsetstrokecolor{dialinecolor}
\definecolor{dialinecolor}{rgb}{1.000000, 1.000000, 1.000000}
\pgfsetfillcolor{dialinecolor}
\definecolor{dialinecolor}{rgb}{1.000000, 1.000000, 1.000000}
\pgfsetfillcolor{dialinecolor}
\pgfpathellipse{\pgfpoint{10.000000\du}{11.000000\du}}{\pgfpoint{1.723870\du}{0\du}}{\pgfpoint{0\du}{1.723870\du}}
\pgfusepath{fill}
\pgfsetlinewidth{0.100000\du}
\pgfsetdash{}{0pt}
\pgfsetdash{}{0pt}
\pgfsetmiterjoin
\definecolor{dialinecolor}{rgb}{0.000000, 0.000000, 0.000000}
\pgfsetstrokecolor{dialinecolor}
\pgfpathellipse{\pgfpoint{10.000000\du}{11.000000\du}}{\pgfpoint{1.723870\du}{0\du}}{\pgfpoint{0\du}{1.723870\du}}
\pgfusepath{stroke}
% setfont left to latex
\definecolor{dialinecolor}{rgb}{0.000000, 0.000000, 0.000000}
\pgfsetstrokecolor{dialinecolor}
\node at (10.000000\du,11.195000\du){CLIENT};
% setfont left to latex
\definecolor{dialinecolor}{rgb}{0.000000, 0.000000, 0.000000}
\pgfsetstrokecolor{dialinecolor}
\node[anchor=west] at (10.000000\du,11.000000\du){};
\definecolor{dialinecolor}{rgb}{1.000000, 1.000000, 1.000000}
\pgfsetfillcolor{dialinecolor}
\pgfpathellipse{\pgfpoint{25.008494\du}{10.996126\du}}{\pgfpoint{1.836494\du}{0\du}}{\pgfpoint{0\du}{1.813886\du}}
\pgfusepath{fill}
\pgfsetlinewidth{0.100000\du}
\pgfsetdash{}{0pt}
\pgfsetdash{}{0pt}
\pgfsetmiterjoin
\definecolor{dialinecolor}{rgb}{0.000000, 0.000000, 0.000000}
\pgfsetstrokecolor{dialinecolor}
\pgfpathellipse{\pgfpoint{25.008494\du}{10.996126\du}}{\pgfpoint{1.836494\du}{0\du}}{\pgfpoint{0\du}{1.813886\du}}
\pgfusepath{stroke}
% setfont left to latex
\definecolor{dialinecolor}{rgb}{0.000000, 0.000000, 0.000000}
\pgfsetstrokecolor{dialinecolor}
\node at (25.008494\du,11.191126\du){SERVER};
\pgfsetlinewidth{0.100000\du}
\pgfsetdash{}{0pt}
\pgfsetdash{}{0pt}
\pgfsetbuttcap
{
\definecolor{dialinecolor}{rgb}{0.000000, 0.000000, 0.000000}
\pgfsetfillcolor{dialinecolor}
% was here!!!
\pgfsetarrowsend{to}
\definecolor{dialinecolor}{rgb}{0.000000, 0.000000, 0.000000}
\pgfsetstrokecolor{dialinecolor}
\draw (10.000000\du,12.774002\du)--(10.000000\du,40.000000\du);
}
\pgfsetlinewidth{0.100000\du}
\pgfsetdash{}{0pt}
\pgfsetdash{}{0pt}
\pgfsetbuttcap
{
\definecolor{dialinecolor}{rgb}{0.000000, 0.000000, 0.000000}
\pgfsetfillcolor{dialinecolor}
% was here!!!
\pgfsetarrowsend{to}
\definecolor{dialinecolor}{rgb}{0.000000, 0.000000, 0.000000}
\pgfsetstrokecolor{dialinecolor}
\draw (25.007949\du,12.859321\du)--(25.000000\du,40.000000\du);
}
\pgfsetlinewidth{0.100000\du}
\pgfsetdash{}{0pt}
\pgfsetdash{}{0pt}
\pgfsetbuttcap
{
\definecolor{dialinecolor}{rgb}{0.000000, 0.000000, 0.000000}
\pgfsetfillcolor{dialinecolor}
% was here!!!
\pgfsetarrowsend{to}
\definecolor{dialinecolor}{rgb}{0.000000, 0.000000, 0.000000}
\pgfsetstrokecolor{dialinecolor}
\draw (10.000000\du,15.799100\du)--(25.006200\du,18.890600\du);
}
\pgfsetlinewidth{0.100000\du}
\pgfsetdash{}{0pt}
\pgfsetdash{}{0pt}
\pgfsetbuttcap
{
\definecolor{dialinecolor}{rgb}{0.000000, 0.000000, 0.000000}
\pgfsetfillcolor{dialinecolor}
% was here!!!
\pgfsetarrowsend{to}
\definecolor{dialinecolor}{rgb}{0.000000, 0.000000, 0.000000}
\pgfsetstrokecolor{dialinecolor}
\draw (25.005300\du,21.906200\du)--(10.000000\du,24.874400\du);
}
\pgfsetlinewidth{0.100000\du}
\pgfsetdash{}{0pt}
\pgfsetdash{}{0pt}
\pgfsetbuttcap
\pgfsetmiterjoin
\pgfsetlinewidth{0.100000\du}
\pgfsetbuttcap
\pgfsetmiterjoin
\pgfsetdash{}{0pt}
\definecolor{dialinecolor}{rgb}{1.000000, 1.000000, 1.000000}
\pgfsetfillcolor{dialinecolor}
\pgfpathmoveto{\pgfpoint{13.558175\du}{16.000000\du}}
\pgfpathlineto{\pgfpoint{19.625675\du}{16.000000\du}}
\pgfpathcurveto{\pgfpoint{20.463422\du}{16.000000\du}}{\pgfpoint{21.142550\du}{16.447715\du}}{\pgfpoint{21.142550\du}{17.000000\du}}
\pgfpathcurveto{\pgfpoint{21.142550\du}{17.552285\du}}{\pgfpoint{20.463422\du}{18.000000\du}}{\pgfpoint{19.625675\du}{18.000000\du}}
\pgfpathlineto{\pgfpoint{13.558175\du}{18.000000\du}}
\pgfpathcurveto{\pgfpoint{12.720428\du}{18.000000\du}}{\pgfpoint{12.041300\du}{17.552285\du}}{\pgfpoint{12.041300\du}{17.000000\du}}
\pgfpathcurveto{\pgfpoint{12.041300\du}{16.447715\du}}{\pgfpoint{12.720428\du}{16.000000\du}}{\pgfpoint{13.558175\du}{16.000000\du}}
\pgfusepath{fill}
\definecolor{dialinecolor}{rgb}{0.000000, 0.000000, 0.000000}
\pgfsetstrokecolor{dialinecolor}
\pgfpathmoveto{\pgfpoint{13.558175\du}{16.000000\du}}
\pgfpathlineto{\pgfpoint{19.625675\du}{16.000000\du}}
\pgfpathcurveto{\pgfpoint{20.463422\du}{16.000000\du}}{\pgfpoint{21.142550\du}{16.447715\du}}{\pgfpoint{21.142550\du}{17.000000\du}}
\pgfpathcurveto{\pgfpoint{21.142550\du}{17.552285\du}}{\pgfpoint{20.463422\du}{18.000000\du}}{\pgfpoint{19.625675\du}{18.000000\du}}
\pgfpathlineto{\pgfpoint{13.558175\du}{18.000000\du}}
\pgfpathcurveto{\pgfpoint{12.720428\du}{18.000000\du}}{\pgfpoint{12.041300\du}{17.552285\du}}{\pgfpoint{12.041300\du}{17.000000\du}}
\pgfpathcurveto{\pgfpoint{12.041300\du}{16.447715\du}}{\pgfpoint{12.720428\du}{16.000000\du}}{\pgfpoint{13.558175\du}{16.000000\du}}
\pgfusepath{stroke}
% setfont left to latex
\definecolor{dialinecolor}{rgb}{0.000000, 0.000000, 0.000000}
\pgfsetstrokecolor{dialinecolor}
\node at (16.591925\du,17.200000\du){\textbf{1}. $Order$};
\pgfsetlinewidth{0.100000\du}
\pgfsetdash{}{0pt}
\pgfsetdash{}{0pt}
\pgfsetbuttcap
\pgfsetmiterjoin
\pgfsetlinewidth{0.100000\du}
\pgfsetbuttcap
\pgfsetmiterjoin
\pgfsetdash{}{0pt}
\definecolor{dialinecolor}{rgb}{1.000000, 1.000000, 1.000000}
\pgfsetfillcolor{dialinecolor}
\pgfpathmoveto{\pgfpoint{12.990000\du}{22.000000\du}}
\pgfpathlineto{\pgfpoint{20.950000\du}{22.000000\du}}
\pgfpathcurveto{\pgfpoint{22.049047\du}{22.000000\du}}{\pgfpoint{22.940000\du}{22.447715\du}}{\pgfpoint{22.940000\du}{23.000000\du}}
\pgfpathcurveto{\pgfpoint{22.940000\du}{23.552285\du}}{\pgfpoint{22.049047\du}{24.000000\du}}{\pgfpoint{20.950000\du}{24.000000\du}}
\pgfpathlineto{\pgfpoint{12.990000\du}{24.000000\du}}
\pgfpathcurveto{\pgfpoint{11.890953\du}{24.000000\du}}{\pgfpoint{11.000000\du}{23.552285\du}}{\pgfpoint{11.000000\du}{23.000000\du}}
\pgfpathcurveto{\pgfpoint{11.000000\du}{22.447715\du}}{\pgfpoint{11.890953\du}{22.000000\du}}{\pgfpoint{12.990000\du}{22.000000\du}}
\pgfusepath{fill}
\definecolor{dialinecolor}{rgb}{0.000000, 0.000000, 0.000000}
\pgfsetstrokecolor{dialinecolor}
\pgfpathmoveto{\pgfpoint{12.990000\du}{22.000000\du}}
\pgfpathlineto{\pgfpoint{20.950000\du}{22.000000\du}}
\pgfpathcurveto{\pgfpoint{22.049047\du}{22.000000\du}}{\pgfpoint{22.940000\du}{22.447715\du}}{\pgfpoint{22.940000\du}{23.000000\du}}
\pgfpathcurveto{\pgfpoint{22.940000\du}{23.552285\du}}{\pgfpoint{22.049047\du}{24.000000\du}}{\pgfpoint{20.950000\du}{24.000000\du}}
\pgfpathlineto{\pgfpoint{12.990000\du}{24.000000\du}}
\pgfpathcurveto{\pgfpoint{11.890953\du}{24.000000\du}}{\pgfpoint{11.000000\du}{23.552285\du}}{\pgfpoint{11.000000\du}{23.000000\du}}
\pgfpathcurveto{\pgfpoint{11.000000\du}{22.447715\du}}{\pgfpoint{11.890953\du}{22.000000\du}}{\pgfpoint{12.990000\du}{22.000000\du}}
\pgfusepath{stroke}
% setfont left to latex
\definecolor{dialinecolor}{rgb}{0.000000, 0.000000, 0.000000}
\pgfsetstrokecolor{dialinecolor}
\node at (16.970000\du,23.200000\du){\textbf{2}.$Coffee$};
\pgfsetlinewidth{0.100000\du}
\pgfsetdash{{1.000000\du}{1.000000\du}}{0\du}
\pgfsetdash{{1.000000\du}{1.000000\du}}{0\du}
\pgfsetbuttcap
{
\definecolor{dialinecolor}{rgb}{0.000000, 0.000000, 0.000000}
\pgfsetfillcolor{dialinecolor}
% was here!!!
\definecolor{dialinecolor}{rgb}{0.000000, 0.000000, 0.000000}
\pgfsetstrokecolor{dialinecolor}
\draw (9.010000\du,25.950000\du)--(36.000000\du,26.000000\du);
}
\definecolor{dialinecolor}{rgb}{1.000000, 1.000000, 1.000000}
\pgfsetfillcolor{dialinecolor}
\fill (0.000000\du,25.000000\du)--(0.000000\du,26.900000\du)--(9.010000\du,26.900000\du)--(9.010000\du,25.000000\du)--cycle;
\pgfsetlinewidth{0.100000\du}
\pgfsetdash{}{0pt}
\pgfsetdash{}{0pt}
\pgfsetmiterjoin
\definecolor{dialinecolor}{rgb}{0.000000, 0.000000, 0.000000}
\pgfsetstrokecolor{dialinecolor}
\draw (0.000000\du,25.000000\du)--(0.000000\du,26.900000\du)--(9.010000\du,26.900000\du)--(9.010000\du,25.000000\du)--cycle;
% setfont left to latex
\definecolor{dialinecolor}{rgb}{0.000000, 0.000000, 0.000000}
\pgfsetstrokecolor{dialinecolor}
\node at (4.505000\du,26.145000\du){If $Flavour$ is not enough};
\pgfsetlinewidth{0.100000\du}
\pgfsetdash{}{0pt}
\pgfsetdash{}{0pt}
\pgfsetbuttcap
{
\definecolor{dialinecolor}{rgb}{0.000000, 0.000000, 0.000000}
\pgfsetfillcolor{dialinecolor}
% was here!!!
\pgfsetarrowsend{to}
\definecolor{dialinecolor}{rgb}{0.000000, 0.000000, 0.000000}
\pgfsetstrokecolor{dialinecolor}
\draw (10.000000\du,27.899600\du)--(25.002700\du,30.953100\du);
}
\pgfsetlinewidth{0.100000\du}
\pgfsetdash{}{0pt}
\pgfsetdash{}{0pt}
\pgfsetbuttcap
\pgfsetmiterjoin
\pgfsetlinewidth{0.100000\du}
\pgfsetbuttcap
\pgfsetmiterjoin
\pgfsetdash{}{0pt}
\definecolor{dialinecolor}{rgb}{1.000000, 1.000000, 1.000000}
\pgfsetfillcolor{dialinecolor}
\pgfpathmoveto{\pgfpoint{14.065625\du}{28.000000\du}}
\pgfpathlineto{\pgfpoint{20.158125\du}{28.000000\du}}
\pgfpathcurveto{\pgfpoint{20.999324\du}{28.000000\du}}{\pgfpoint{21.681250\du}{28.447715\du}}{\pgfpoint{21.681250\du}{29.000000\du}}
\pgfpathcurveto{\pgfpoint{21.681250\du}{29.552285\du}}{\pgfpoint{20.999324\du}{30.000000\du}}{\pgfpoint{20.158125\du}{30.000000\du}}
\pgfpathlineto{\pgfpoint{14.065625\du}{30.000000\du}}
\pgfpathcurveto{\pgfpoint{13.224426\du}{30.000000\du}}{\pgfpoint{12.542500\du}{29.552285\du}}{\pgfpoint{12.542500\du}{29.000000\du}}
\pgfpathcurveto{\pgfpoint{12.542500\du}{28.447715\du}}{\pgfpoint{13.224426\du}{28.000000\du}}{\pgfpoint{14.065625\du}{28.000000\du}}
\pgfusepath{fill}
\definecolor{dialinecolor}{rgb}{0.000000, 0.000000, 0.000000}
\pgfsetstrokecolor{dialinecolor}
\pgfpathmoveto{\pgfpoint{14.065625\du}{28.000000\du}}
\pgfpathlineto{\pgfpoint{20.158125\du}{28.000000\du}}
\pgfpathcurveto{\pgfpoint{20.999324\du}{28.000000\du}}{\pgfpoint{21.681250\du}{28.447715\du}}{\pgfpoint{21.681250\du}{29.000000\du}}
\pgfpathcurveto{\pgfpoint{21.681250\du}{29.552285\du}}{\pgfpoint{20.999324\du}{30.000000\du}}{\pgfpoint{20.158125\du}{30.000000\du}}
\pgfpathlineto{\pgfpoint{14.065625\du}{30.000000\du}}
\pgfpathcurveto{\pgfpoint{13.224426\du}{30.000000\du}}{\pgfpoint{12.542500\du}{29.552285\du}}{\pgfpoint{12.542500\du}{29.000000\du}}
\pgfpathcurveto{\pgfpoint{12.542500\du}{28.447715\du}}{\pgfpoint{13.224426\du}{28.000000\du}}{\pgfpoint{14.065625\du}{28.000000\du}}
\pgfusepath{stroke}
% setfont left to latex
\definecolor{dialinecolor}{rgb}{0.000000, 0.000000, 0.000000}
\pgfsetstrokecolor{dialinecolor}
\node at (17.111875\du,29.200000\du){\textbf{3}.$FlavourRequest$};
\pgfsetlinewidth{0.100000\du}
\pgfsetdash{{1.000000\du}{1.000000\du}}{0\du}
\pgfsetdash{{1.000000\du}{1.000000\du}}{0\du}
\pgfsetbuttcap
{
\definecolor{dialinecolor}{rgb}{0.000000, 0.000000, 0.000000}
\pgfsetfillcolor{dialinecolor}
% was here!!!
\definecolor{dialinecolor}{rgb}{0.000000, 0.000000, 0.000000}
\pgfsetstrokecolor{dialinecolor}
\draw (1.000000\du,38.000000\du)--(36.000000\du,38.000000\du);
}
\pgfsetlinewidth{0.100000\du}
\pgfsetdash{}{0pt}
\pgfsetdash{}{0pt}
\pgfsetbuttcap
{
\definecolor{dialinecolor}{rgb}{0.000000, 0.000000, 0.000000}
\pgfsetfillcolor{dialinecolor}
% was here!!!
\pgfsetarrowsend{to}
\definecolor{dialinecolor}{rgb}{0.000000, 0.000000, 0.000000}
\pgfsetstrokecolor{dialinecolor}
\draw (25.001800\du,33.968700\du)--(10.000000\du,36.974900\du);
}
\pgfsetlinewidth{0.100000\du}
\pgfsetdash{}{0pt}
\pgfsetdash{}{0pt}
\pgfsetbuttcap
\pgfsetmiterjoin
\pgfsetlinewidth{0.100000\du}
\pgfsetbuttcap
\pgfsetmiterjoin
\pgfsetdash{}{0pt}
\definecolor{dialinecolor}{rgb}{1.000000, 1.000000, 1.000000}
\pgfsetfillcolor{dialinecolor}
\pgfpathmoveto{\pgfpoint{13.779375\du}{34.000000\du}}
\pgfpathlineto{\pgfpoint{20.236875\du}{34.000000\du}}
\pgfpathcurveto{\pgfpoint{21.128470\du}{34.000000\du}}{\pgfpoint{21.851250\du}{34.447715\du}}{\pgfpoint{21.851250\du}{35.000000\du}}
\pgfpathcurveto{\pgfpoint{21.851250\du}{35.552285\du}}{\pgfpoint{21.128470\du}{36.000000\du}}{\pgfpoint{20.236875\du}{36.000000\du}}
\pgfpathlineto{\pgfpoint{13.779375\du}{36.000000\du}}
\pgfpathcurveto{\pgfpoint{12.887780\du}{36.000000\du}}{\pgfpoint{12.165000\du}{35.552285\du}}{\pgfpoint{12.165000\du}{35.000000\du}}
\pgfpathcurveto{\pgfpoint{12.165000\du}{34.447715\du}}{\pgfpoint{12.887780\du}{34.000000\du}}{\pgfpoint{13.779375\du}{34.000000\du}}
\pgfusepath{fill}
\definecolor{dialinecolor}{rgb}{0.000000, 0.000000, 0.000000}
\pgfsetstrokecolor{dialinecolor}
\pgfpathmoveto{\pgfpoint{13.779375\du}{34.000000\du}}
\pgfpathlineto{\pgfpoint{20.236875\du}{34.000000\du}}
\pgfpathcurveto{\pgfpoint{21.128470\du}{34.000000\du}}{\pgfpoint{21.851250\du}{34.447715\du}}{\pgfpoint{21.851250\du}{35.000000\du}}
\pgfpathcurveto{\pgfpoint{21.851250\du}{35.552285\du}}{\pgfpoint{21.128470\du}{36.000000\du}}{\pgfpoint{20.236875\du}{36.000000\du}}
\pgfpathlineto{\pgfpoint{13.779375\du}{36.000000\du}}
\pgfpathcurveto{\pgfpoint{12.887780\du}{36.000000\du}}{\pgfpoint{12.165000\du}{35.552285\du}}{\pgfpoint{12.165000\du}{35.000000\du}}
\pgfpathcurveto{\pgfpoint{12.165000\du}{34.447715\du}}{\pgfpoint{12.887780\du}{34.000000\du}}{\pgfpoint{13.779375\du}{34.000000\du}}
\pgfusepath{stroke}
% setfont left to latex
\definecolor{dialinecolor}{rgb}{0.000000, 0.000000, 0.000000}
\pgfsetstrokecolor{dialinecolor}
\node at (17.008125\du,35.200000\du){\textbf{4}.$FlavourResponse$};
\pgfsetlinewidth{0.100000\du}
\pgfsetdash{{1.000000\du}{1.000000\du}}{0\du}
\pgfsetdash{{1.000000\du}{1.000000\du}}{0\du}
\pgfsetbuttcap
{
\definecolor{dialinecolor}{rgb}{0.000000, 0.000000, 0.000000}
\pgfsetfillcolor{dialinecolor}
% was here!!!
\pgfsetarrowsstart{to}
\pgfsetarrowsend{to}
\definecolor{dialinecolor}{rgb}{0.000000, 0.000000, 0.000000}
\pgfsetstrokecolor{dialinecolor}
\draw (5.000000\du,27.000000\du)--(5.000000\du,38.000000\du);
}
\end{tikzpicture}
}
\end{figure}

\begin{description}
\item[1] As an initiation of a conversation, CLIENT randomly picks a string $Order \in COFFEE$ and sends it to SERVER.

\item[2] Upon receiving an \textit{Order}, SERVER replies with a string $\{Order || Flavour\}$ where $Flavour \in ADDITIVE$ and $||$ represents concatenation. If $Order = \text{"ESPRESSO"}$ then the degrees of both SUGAR and MILK of \textit{Flavour} are set to $0$.

\item[3] CLIENT randomly generates a SUGAR requirement $r_* \in [0, 3]$ and a MILK requirement $r_@ \in [0,3]$. Then it scans the reply from \textbf{2} and computes its degrees of SUGAR and MILK. If any of the degrees does not  met the requirements, i.e. $n_* < r_*$ and/or $n_@ < r_@$, then CLIENT sends a $ FlavourRequest = \{"FLAVOUR"||\{SUGAR\}^{\max({r_* - n_*,0})} || MILK^{\max(r_@ -  n_@, 0)} \} $.

\item[4] If SERVER receives a $FlavourRequest$, it echoes back $FlavourRequest$ as its $FlavourResponse$, i.e. $FlvaourResponse = FlavourRequest$.	
\end{description}

Note that the $FlavourRequest$ and $FlavourResponse$ packets are probabilistic in a Leaky-Coffee Session.

In this experimental implementation, CLIENT will be putted into sleep after each session for a random period of 5 to 15 seconds.

\begin{example}
An example with $FlavourRequest$ and $FlavourResponse$\ref{Fig: Leaky-Coffee Example1}:

\begin{figure}[H] \label{Fig: Leaky-Coffee Example1}
\caption{Example: A Leaky-Coffee session with $FlavourRequest$ and $FlavourResponse$}
\centering
\resizebox{6cm}{!}
{% Graphic for TeX using PGF
% Title: /home/yy12135/Writings/First-year-review_20150422/Pics/LeakyCoffee_example1.dia
% Creator: Dia v0.97.2
% CreationDate: Tue Mar  3 11:33:54 2015
% For: yy12135
% \usepackage{tikz}
% The following commands are not supported in PSTricks at present
% We define them conditionally, so when they are implemented,
% this pgf file will use them.
\ifx\du\undefined
  \newlength{\du}
\fi
\setlength{\du}{15\unitlength}
\begin{tikzpicture}
\pgftransformxscale{1.000000}
\pgftransformyscale{-1.000000}
\definecolor{dialinecolor}{rgb}{0.000000, 0.000000, 0.000000}
\pgfsetstrokecolor{dialinecolor}
\definecolor{dialinecolor}{rgb}{1.000000, 1.000000, 1.000000}
\pgfsetfillcolor{dialinecolor}
\definecolor{dialinecolor}{rgb}{1.000000, 1.000000, 1.000000}
\pgfsetfillcolor{dialinecolor}
\pgfpathellipse{\pgfpoint{27.000000\du}{12.000000\du}}{\pgfpoint{2.000000\du}{0\du}}{\pgfpoint{0\du}{2.000000\du}}
\pgfusepath{fill}
\pgfsetlinewidth{0.100000\du}
\pgfsetdash{}{0pt}
\pgfsetdash{}{0pt}
\pgfsetmiterjoin
\definecolor{dialinecolor}{rgb}{0.000000, 0.000000, 0.000000}
\pgfsetstrokecolor{dialinecolor}
\pgfpathellipse{\pgfpoint{27.000000\du}{12.000000\du}}{\pgfpoint{2.000000\du}{0\du}}{\pgfpoint{0\du}{2.000000\du}}
\pgfusepath{stroke}
% setfont left to latex
\definecolor{dialinecolor}{rgb}{0.000000, 0.000000, 0.000000}
\pgfsetstrokecolor{dialinecolor}
\node at (27.000000\du,12.195000\du){CLIENT};
\definecolor{dialinecolor}{rgb}{1.000000, 1.000000, 1.000000}
\pgfsetfillcolor{dialinecolor}
\pgfpathellipse{\pgfpoint{40.051131\du}{12.044456\du}}{\pgfpoint{2.051131\du}{0\du}}{\pgfpoint{0\du}{2.044456\du}}
\pgfusepath{fill}
\pgfsetlinewidth{0.100000\du}
\pgfsetdash{}{0pt}
\pgfsetdash{}{0pt}
\pgfsetmiterjoin
\definecolor{dialinecolor}{rgb}{0.000000, 0.000000, 0.000000}
\pgfsetstrokecolor{dialinecolor}
\pgfpathellipse{\pgfpoint{40.051131\du}{12.044456\du}}{\pgfpoint{2.051131\du}{0\du}}{\pgfpoint{0\du}{2.044456\du}}
\pgfusepath{stroke}
% setfont left to latex
\definecolor{dialinecolor}{rgb}{0.000000, 0.000000, 0.000000}
\pgfsetstrokecolor{dialinecolor}
\node at (40.051131\du,12.239456\du){SERVER};
\pgfsetlinewidth{0.100000\du}
\pgfsetdash{}{0pt}
\pgfsetdash{}{0pt}
\pgfsetbuttcap
{
\definecolor{dialinecolor}{rgb}{0.000000, 0.000000, 0.000000}
\pgfsetfillcolor{dialinecolor}
% was here!!!
\pgfsetarrowsend{to}
\definecolor{dialinecolor}{rgb}{0.000000, 0.000000, 0.000000}
\pgfsetstrokecolor{dialinecolor}
\draw (27.000000\du,14.000000\du)--(27.000000\du,40.000000\du);
}
\pgfsetlinewidth{0.100000\du}
\pgfsetdash{}{0pt}
\pgfsetdash{}{0pt}
\pgfsetbuttcap
{
\definecolor{dialinecolor}{rgb}{0.000000, 0.000000, 0.000000}
\pgfsetfillcolor{dialinecolor}
% was here!!!
\pgfsetarrowsend{to}
\definecolor{dialinecolor}{rgb}{0.000000, 0.000000, 0.000000}
\pgfsetstrokecolor{dialinecolor}
\draw (40.051131\du,14.088911\du)--(40.000000\du,40.000000\du);
}
\pgfsetlinewidth{0.100000\du}
\pgfsetdash{}{0pt}
\pgfsetdash{}{0pt}
\pgfsetbuttcap
{
\definecolor{dialinecolor}{rgb}{0.000000, 0.000000, 0.000000}
\pgfsetfillcolor{dialinecolor}
% was here!!!
\pgfsetarrowsend{to}
\definecolor{dialinecolor}{rgb}{0.000000, 0.000000, 0.000000}
\pgfsetstrokecolor{dialinecolor}
\draw (27.000000\du,16.888889\du)--(40.128953\du,20.642722\du);
}
\pgfsetlinewidth{0.100000\du}
\pgfsetdash{}{0pt}
\pgfsetdash{}{0pt}
\pgfsetbuttcap
{
\definecolor{dialinecolor}{rgb}{0.000000, 0.000000, 0.000000}
\pgfsetfillcolor{dialinecolor}
% was here!!!
\pgfsetarrowsend{to}
\definecolor{dialinecolor}{rgb}{0.000000, 0.000000, 0.000000}
\pgfsetstrokecolor{dialinecolor}
\draw (40.034087\du,22.725941\du)--(27.000000\du,25.555556\du);
}
\pgfsetlinewidth{0.100000\du}
\pgfsetdash{}{0pt}
\pgfsetdash{}{0pt}
\pgfsetbuttcap
{
\definecolor{dialinecolor}{rgb}{0.000000, 0.000000, 0.000000}
\pgfsetfillcolor{dialinecolor}
% was here!!!
\pgfsetarrowsend{to}
\definecolor{dialinecolor}{rgb}{0.000000, 0.000000, 0.000000}
\pgfsetstrokecolor{dialinecolor}
\draw (27.000000\du,28.444444\du)--(40.017044\du,31.362970\du);
}
% setfont left to latex
\definecolor{dialinecolor}{rgb}{0.000000, 0.000000, 0.000000}
\pgfsetstrokecolor{dialinecolor}
\node[anchor=west] at (33.517044\du,24.140748\du){};
% setfont left to latex
\definecolor{dialinecolor}{rgb}{0.000000, 0.000000, 0.000000}
\pgfsetstrokecolor{dialinecolor}
\node[anchor=west] at (32.000000\du,20.000000\du){};
% setfont left to latex
\definecolor{dialinecolor}{rgb}{0.000000, 0.000000, 0.000000}
\pgfsetstrokecolor{dialinecolor}
\node[anchor=west] at (33.508522\du,29.903707\du){};
\pgfsetlinewidth{0.100000\du}
\pgfsetdash{}{0pt}
\pgfsetdash{}{0pt}
\pgfsetbuttcap
{
\definecolor{dialinecolor}{rgb}{0.000000, 0.000000, 0.000000}
\pgfsetfillcolor{dialinecolor}
% was here!!!
\pgfsetarrowsend{to}
\definecolor{dialinecolor}{rgb}{0.000000, 0.000000, 0.000000}
\pgfsetstrokecolor{dialinecolor}
\draw (40.011362\du,34.241980\du)--(27.000000\du,37.111111\du);
}
\definecolor{dialinecolor}{rgb}{1.000000, 1.000000, 1.000000}
\pgfsetfillcolor{dialinecolor}
\fill (31.351581\du,18.005049\du)--(31.351581\du,19.905049\du)--(35.466581\du,19.905049\du)--(35.466581\du,18.005049\du)--cycle;
\pgfsetlinewidth{0.100000\du}
\pgfsetdash{}{0pt}
\pgfsetdash{}{0pt}
\pgfsetmiterjoin
\definecolor{dialinecolor}{rgb}{0.000000, 0.000000, 0.000000}
\pgfsetstrokecolor{dialinecolor}
\draw (31.351581\du,18.005049\du)--(31.351581\du,19.905049\du)--(35.466581\du,19.905049\du)--(35.466581\du,18.005049\du)--cycle;
% setfont left to latex
\definecolor{dialinecolor}{rgb}{0.000000, 0.000000, 0.000000}
\pgfsetstrokecolor{dialinecolor}
\node at (33.409081\du,19.150049\du){“MOCHA”};
\definecolor{dialinecolor}{rgb}{1.000000, 1.000000, 1.000000}
\pgfsetfillcolor{dialinecolor}
\fill (30.842846\du,23.045766\du)--(30.842846\du,24.945766\du)--(35.917846\du,24.945766\du)--(35.917846\du,23.045766\du)--cycle;
\pgfsetlinewidth{0.100000\du}
\pgfsetdash{}{0pt}
\pgfsetdash{}{0pt}
\pgfsetmiterjoin
\definecolor{dialinecolor}{rgb}{0.000000, 0.000000, 0.000000}
\pgfsetstrokecolor{dialinecolor}
\draw (30.842846\du,23.045766\du)--(30.842846\du,24.945766\du)--(35.917846\du,24.945766\du)--(35.917846\du,23.045766\du)--cycle;
% setfont left to latex
\definecolor{dialinecolor}{rgb}{0.000000, 0.000000, 0.000000}
\pgfsetstrokecolor{dialinecolor}
\node at (33.380346\du,24.190766\du){“MOCHA*@”};
\definecolor{dialinecolor}{rgb}{1.000000, 1.000000, 1.000000}
\pgfsetfillcolor{dialinecolor}
\fill (30.418621\du,29.238167\du)--(30.418621\du,31.138167\du)--(36.731121\du,31.138167\du)--(36.731121\du,29.238167\du)--cycle;
\pgfsetlinewidth{0.100000\du}
\pgfsetdash{}{0pt}
\pgfsetdash{}{0pt}
\pgfsetmiterjoin
\definecolor{dialinecolor}{rgb}{0.000000, 0.000000, 0.000000}
\pgfsetstrokecolor{dialinecolor}
\draw (30.418621\du,29.238167\du)--(30.418621\du,31.138167\du)--(36.731121\du,31.138167\du)--(36.731121\du,29.238167\du)--cycle;
% setfont left to latex
\definecolor{dialinecolor}{rgb}{0.000000, 0.000000, 0.000000}
\pgfsetstrokecolor{dialinecolor}
\node at (33.574871\du,30.383167\du){“FLAVOUR**@”};
\definecolor{dialinecolor}{rgb}{1.000000, 1.000000, 1.000000}
\pgfsetfillcolor{dialinecolor}
\fill (30.598621\du,34.620045\du)--(30.598621\du,36.520045\du)--(36.551121\du,36.520045\du)--(36.551121\du,34.620045\du)--cycle;
\pgfsetlinewidth{0.100000\du}
\pgfsetdash{}{0pt}
\pgfsetdash{}{0pt}
\pgfsetmiterjoin
\definecolor{dialinecolor}{rgb}{0.000000, 0.000000, 0.000000}
\pgfsetstrokecolor{dialinecolor}
\draw (30.598621\du,34.620045\du)--(30.598621\du,36.520045\du)--(36.551121\du,36.520045\du)--(36.551121\du,34.620045\du)--cycle;
% setfont left to latex
\definecolor{dialinecolor}{rgb}{0.000000, 0.000000, 0.000000}
\pgfsetstrokecolor{dialinecolor}
\node at (33.574871\du,35.765045\du){“FLAVOUR**@”};
\end{tikzpicture}
}
\end{figure}

In this example, CLIENT first sends an $Order$ “MOCHA”. SERVER then replies with “MOCHA*@” which implies both the SUGAR degree and MILK degree are $1$. CLIENT randomly generates a SUGAR requirement $3$ and MILK requirement $2$ and then sends a $FlavourRequest$ to request the shorted SUGAR and MILK. SERVER finally response with the requested ADDITIVE.
\end{example}

\begin{example}
Another example without $FlavourRequest$ and $FlavourResponse$:

\begin{figure}[H] \label{Fig: Leaky-Coffee Example2}
\caption{Example: A Leaky-Coffee session without $FlavourRequest$ and $FlavourResponse$}
\centering
\resizebox{6cm}{!}
{% Graphic for TeX using PGF
% Title: /home/yy12135/Writings/First-year-review_20150422/Pics/LeakyCoffee_example2.dia
% Creator: Dia v0.97.2
% CreationDate: Tue Mar  3 13:37:23 2015
% For: yy12135
% \usepackage{tikz}
% The following commands are not supported in PSTricks at present
% We define them conditionally, so when they are implemented,
% this pgf file will use them.
\ifx\du\undefined
  \newlength{\du}
\fi
\setlength{\du}{15\unitlength}
\begin{tikzpicture}
\pgftransformxscale{1.000000}
\pgftransformyscale{-1.000000}
\definecolor{dialinecolor}{rgb}{0.000000, 0.000000, 0.000000}
\pgfsetstrokecolor{dialinecolor}
\definecolor{dialinecolor}{rgb}{1.000000, 1.000000, 1.000000}
\pgfsetfillcolor{dialinecolor}
\definecolor{dialinecolor}{rgb}{1.000000, 1.000000, 1.000000}
\pgfsetfillcolor{dialinecolor}
\pgfpathellipse{\pgfpoint{31.000000\du}{12.000000\du}}{\pgfpoint{2.000000\du}{0\du}}{\pgfpoint{0\du}{2.000000\du}}
\pgfusepath{fill}
\pgfsetlinewidth{0.100000\du}
\pgfsetdash{}{0pt}
\pgfsetdash{}{0pt}
\pgfsetmiterjoin
\definecolor{dialinecolor}{rgb}{0.000000, 0.000000, 0.000000}
\pgfsetstrokecolor{dialinecolor}
\pgfpathellipse{\pgfpoint{31.000000\du}{12.000000\du}}{\pgfpoint{2.000000\du}{0\du}}{\pgfpoint{0\du}{2.000000\du}}
\pgfusepath{stroke}
% setfont left to latex
\definecolor{dialinecolor}{rgb}{0.000000, 0.000000, 0.000000}
\pgfsetstrokecolor{dialinecolor}
\node at (31.000000\du,12.195000\du){CLIENT};
\definecolor{dialinecolor}{rgb}{1.000000, 1.000000, 1.000000}
\pgfsetfillcolor{dialinecolor}
\pgfpathellipse{\pgfpoint{40.051131\du}{12.044456\du}}{\pgfpoint{2.051131\du}{0\du}}{\pgfpoint{0\du}{2.044456\du}}
\pgfusepath{fill}
\pgfsetlinewidth{0.100000\du}
\pgfsetdash{}{0pt}
\pgfsetdash{}{0pt}
\pgfsetmiterjoin
\definecolor{dialinecolor}{rgb}{0.000000, 0.000000, 0.000000}
\pgfsetstrokecolor{dialinecolor}
\pgfpathellipse{\pgfpoint{40.051131\du}{12.044456\du}}{\pgfpoint{2.051131\du}{0\du}}{\pgfpoint{0\du}{2.044456\du}}
\pgfusepath{stroke}
% setfont left to latex
\definecolor{dialinecolor}{rgb}{0.000000, 0.000000, 0.000000}
\pgfsetstrokecolor{dialinecolor}
\node at (40.051131\du,12.239456\du){SERVER};
\pgfsetlinewidth{0.100000\du}
\pgfsetdash{}{0pt}
\pgfsetdash{}{0pt}
\pgfsetbuttcap
{
\definecolor{dialinecolor}{rgb}{0.000000, 0.000000, 0.000000}
\pgfsetfillcolor{dialinecolor}
% was here!!!
\pgfsetarrowsend{to}
\definecolor{dialinecolor}{rgb}{0.000000, 0.000000, 0.000000}
\pgfsetstrokecolor{dialinecolor}
\draw (31.000000\du,14.000000\du)--(31.000000\du,24.000000\du);
}
\pgfsetlinewidth{0.100000\du}
\pgfsetdash{}{0pt}
\pgfsetdash{}{0pt}
\pgfsetbuttcap
{
\definecolor{dialinecolor}{rgb}{0.000000, 0.000000, 0.000000}
\pgfsetfillcolor{dialinecolor}
% was here!!!
\pgfsetarrowsend{to}
\definecolor{dialinecolor}{rgb}{0.000000, 0.000000, 0.000000}
\pgfsetstrokecolor{dialinecolor}
\draw (40.051100\du,14.088900\du)--(40.000000\du,24.000000\du);
}
\pgfsetlinewidth{0.100000\du}
\pgfsetdash{}{0pt}
\pgfsetdash{}{0pt}
\pgfsetbuttcap
{
\definecolor{dialinecolor}{rgb}{0.000000, 0.000000, 0.000000}
\pgfsetfillcolor{dialinecolor}
% was here!!!
\pgfsetarrowsend{to}
\definecolor{dialinecolor}{rgb}{0.000000, 0.000000, 0.000000}
\pgfsetstrokecolor{dialinecolor}
\draw (31.000000\du,16.000000\du)--(40.030660\du,18.053340\du);
}
\pgfsetlinewidth{0.100000\du}
\pgfsetdash{}{0pt}
\pgfsetdash{}{0pt}
\pgfsetbuttcap
{
\definecolor{dialinecolor}{rgb}{0.000000, 0.000000, 0.000000}
\pgfsetfillcolor{dialinecolor}
% was here!!!
\pgfsetarrowsend{to}
\definecolor{dialinecolor}{rgb}{0.000000, 0.000000, 0.000000}
\pgfsetstrokecolor{dialinecolor}
\draw (40.020440\du,20.035560\du)--(31.000000\du,22.000000\du);
}
% setfont left to latex
\definecolor{dialinecolor}{rgb}{0.000000, 0.000000, 0.000000}
\pgfsetstrokecolor{dialinecolor}
\node[anchor=west] at (35.510220\du,21.017780\du){};
% setfont left to latex
\definecolor{dialinecolor}{rgb}{0.000000, 0.000000, 0.000000}
\pgfsetstrokecolor{dialinecolor}
\node[anchor=west] at (32.000000\du,20.000000\du){};
\definecolor{dialinecolor}{rgb}{1.000000, 1.000000, 1.000000}
\pgfsetfillcolor{dialinecolor}
\fill (33.000000\du,16.000000\du)--(33.000000\du,17.900000\du)--(38.057500\du,17.900000\du)--(38.057500\du,16.000000\du)--cycle;
\pgfsetlinewidth{0.100000\du}
\pgfsetdash{}{0pt}
\pgfsetdash{}{0pt}
\pgfsetmiterjoin
\definecolor{dialinecolor}{rgb}{0.000000, 0.000000, 0.000000}
\pgfsetstrokecolor{dialinecolor}
\draw (33.000000\du,16.000000\du)--(33.000000\du,17.900000\du)--(38.057500\du,17.900000\du)--(38.057500\du,16.000000\du)--cycle;
% setfont left to latex
\definecolor{dialinecolor}{rgb}{0.000000, 0.000000, 0.000000}
\pgfsetstrokecolor{dialinecolor}
\node at (35.528750\du,17.145000\du){"ESPRESSO"};
\definecolor{dialinecolor}{rgb}{1.000000, 1.000000, 1.000000}
\pgfsetfillcolor{dialinecolor}
\fill (33.000000\du,20.000000\du)--(33.000000\du,21.900000\du)--(38.075000\du,21.900000\du)--(38.075000\du,20.000000\du)--cycle;
\pgfsetlinewidth{0.100000\du}
\pgfsetdash{}{0pt}
\pgfsetdash{}{0pt}
\pgfsetmiterjoin
\definecolor{dialinecolor}{rgb}{0.000000, 0.000000, 0.000000}
\pgfsetstrokecolor{dialinecolor}
\draw (33.000000\du,20.000000\du)--(33.000000\du,21.900000\du)--(38.075000\du,21.900000\du)--(38.075000\du,20.000000\du)--cycle;
% setfont left to latex
\definecolor{dialinecolor}{rgb}{0.000000, 0.000000, 0.000000}
\pgfsetstrokecolor{dialinecolor}
\node at (35.537500\du,21.145000\du){"ESPRESSO"};
\end{tikzpicture}
}
\end{figure}

This example demonstrates a session initiated with “ESPRESSO” where no ADDITIVE will be added in the reply.
\end{example}

\section{Analysis [to be completed]}
