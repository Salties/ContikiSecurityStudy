\section{Introduction}
Traffic Analysis (TA) is a well studied technique that breaches data confidentiality over Internet, typically targeting protocols including HTTPS\cite{rfc2818} and SSL\cite{rfc6101}/TLS\cite{rfc5246}. This type of attacks works by correlating the encrypted contents to side channel information frequently omitted by cryptographic scheme designs, such as packet length, timing and any other unprotected meta data. Consequently, TA is often associated to Internet privacy violation and Mass Surveillance.

The extensive use of radio communication in exposed environment poses a great security challenge to IoT applications, especially against TA attacks as packets containing critical privacy data, e.g. driving information in Vehicular Ad Hoc Networks(VANETs)\cite{VANET} and daily life data in a smart house, can be easily eavesdropped and analysed by adversaries.

In this paper, we discuss the applicability of TA techniques over IoT application traffic. To be more specifically, we present an attempt of TA attacks on a 6LoWPAN\cite{rfc4944} network powered by Contiki\cite{Contiki} and demonstrate the potential of extracting ``supposedly'' private information from. Although revealing these information does not directly imply any security problem in the setup due to the fact that they are purely designed for experimental purposes, but the result indicates that TA attacks should be concerned when designing secure IoT applications.

%Finally the paper structure
%To be done.

%\subsection{Related Work}
%Literatures about Traffic Analysis...
%Literatures about 6LoWPAN security